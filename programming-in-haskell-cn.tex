%
% TeX模板
%
% 中文支持方案: XeTeX + xeCJK
% author: Tony Bai
% 
% compile: 
%     xelatex programming-in-haskell-cn.tex
%

\documentclass[a4paper,11pt,titlepage]{book}                % 五号字

%
% preamble begin {
%

\usepackage{fontspec}
\usepackage{xunicode} 
\usepackage{xltxtra}                                        % xltxtra include the cmd \XeTeX

%\XeTeXinputencoding "GBK"                                   % 本文件采用GBK编码
\XeTeXlinebreaklocale "zh"
\XeTeXlinebreakskip = 0pt plus 1pt minus 0.1pt

\usepackage{titletoc}                                       % 控制目录
\usepackage{appendix}                                       % 控制附录

\usepackage[colorlinks,
            linkcolor=black,
            citecolor=black]{hyperref}                      % \url

% 页边距设置
\usepackage[top=1.2in,bottom=1.2in,left=1.2in,right=1in]{geometry}

% 封面设置
\title{Haskell程序设计}
\author{著:Graham Hutton\\
        译:Tony Bai\footnote{\url{http://bigwhite.blogbus.com}}}
\date{October, 2010}

% 页眉页脚设置
\usepackage{fancyhdr}
\pagestyle{fancy}
\fancyhf{}                                                  % 清空页眉页脚
\fancyhead[LE,RO]{\thepage}                                 % 偶数页左,奇数页右
\fancyhead[RE]{\leftmark}                                   % 偶数页右
\fancyhead[LO]{\rightmark}                                  % 奇数页左
\fancypagestyle{plain}{
\fancyhf{}                                                  % 重定义plain页面样式
\renewcommand{\headrulewidth}{0pt}
}
\renewcommand\chaptermark[1]{\markboth{\chaptername~ #1}{}}
\renewcommand\sectionmark[1]{\markright{\thesection~ #1}}

% 行距
\renewcommand{\baselinestretch}{1.25}

% 章节设置
\usepackage{titlesec} 
\titleformat{\chapter}{\centering\huge}{第\thechapter{}章}{1em}{\textbf}

% xeCJK设置
\usepackage[slantfont, boldfont, CJKaddspaces]{xeCJK}
\setmainfont{Times New Roman}                                % view the font list through the cmd "fc-list :lang=en"
\setCJKmainfont{SimSun}                                      % view the font list through the cmd "fc-list :lang=zh"
\setCJKfamilyfont{song}{SimSun}
\setsansfont{AR PL UKai CN}

% 重定义设置
\renewcommand{\chaptername}{第{\thechapter}章}
\renewcommand{\contentsname}{目~录}
\renewcommand{\indexname}{索引}
\renewcommand{\listfigurename}{插图目录}
\renewcommand{\listtablename}{表格目录}
\renewcommand{\figurename}{图}
\renewcommand{\tablename}{表}
\renewcommand{\appendixname}{附录}
\renewcommand{\appendixpagename}{附录}
\renewcommand{\appendixtocname}{附录}
%\renewcommand\refname{参考文献}

\usepackage[fleqn]{amsmath}                                 % 

%
% } preamble end
%


%
% body begin {
%

\begin{document}

\maketitle                                                  % 生成title

\chapter*{Haskell程序设计}
\textit{待译}

                                        % half_title
\chapter*{版权声明}
由于尚未获得原作者授权,所以本项目仅用于学习和交流之用,不得用于任何商业用途。
                                         % 版权声明
\chapter*{序}
\addcontentsline{toc}{chapter}{序}
\begin{quotation}
...这个世界上有两种设计软件的方法:一种是使设计尽量的简化,以至于明显没有任何缺陷;而另一种是使设计尽量的复杂,以至于找不到明显的缺陷。第一种方法更加困难。
\begin{flushright}
\textit{ -- Tony Hoare,1980 ACM图灵奖演讲}
\end{flushright}
\end{quotation}
这本书讲述了一种以简单、清晰和优雅为关键目标的编程方法。更具体的说,它是一本使用Haskell语言介绍这种函数式编程方法的入门书籍。

函数式编程与当前大多数编程语言比如Java,C++,C和Visual
Basic所提倡的风格相差很大。特别是当前大多数语言是与底层硬件紧密关联的,在这种意义上,编程的基本思想就是修改存储的值。与此相反,Haskell则提倡一种更为抽象的编程风格,这种风格的基本思想则是将函数应用于参数。正如我们将要看到的,站上更高的层次可以让我们拥有更为简单的程序,并且支持一系列强大的构造和推导程序的新方法。

本书主要面对具备大专水平的学习计算机科学的学生,但也同样广泛适用于那些想了解和学习Haskell编程的读者。你不需要拥有任何编程经验,所有的概念都从基本原理讲起,并伴以精心挑选的例子。

本书使用的Haskell版本是Haskell
98,Haskell语言的标准版本,Haskell设计者花费了15年才最终发布了这个标准。由于这里仅仅是入门引导,所以我们不会去涉及
Haskell语言以及其相关库的所有方面。本书大约有一半内容是专门介绍该语言的主要特点的,另一半则包括了Haskell编程的例子和案例研究。每个章节还包括了一系列的习题以及关于进一步阅读更高级、更专业主题的建议。

本书基于课程材料编写而成,这些课程材料在诺丁汉大学经过了多年的改进和课程测试。通过20学时的授课以及大约40小时的自学、实验室实践和编程作业,你就可以学完本书的大部分内容。然而,你还需要更多的时间详细学习一下后面的一些章节以及一些编程例子。

本书的官方网站提供了一系列辅助资料,包括每个章节的幻灯片和一些扩展例子的Haskell代码。教师还可以通过发送电邮到solutions@cambridge.org得到每个章节练习题的标准答案以及大量带有标准答案的试题集。

\begin{flushleft}
致谢\\
\end{flushleft}

诺丁汉大学编程社团为函数式编程的研究与教学提供了一个极好的环境。感谢学校提供假期让我可以进行本书的写作。感谢所有学生和讲师们关于我的Haskell课程的反馈;
感谢Thorsten Altenkirch, Neil Ghani, Mark Jones (现在在波特兰州), Conor McBrideh和Henrik Nilsson在FOP社区与我进行的关于函数式编程的想法以及如何表达这些想法的交流和讨论。

我还要感谢David Tranah和Dawn Preston在剑桥大学出版社出色的编辑工作; 感谢Mark
Jones的Haskell解释器; 感谢Ralf Hinze和Andres Loh提供的lhs2TeX排版系统;感谢Rik
van Geldrop和Jaap van der Woude关于使用本书草稿的反馈; 感谢Kees van den Broek,
Frank Heitmann和 Bill Tonkin指出的本书的错误;感谢Ian
Bayley和所有匿名评审者极富价值的评论;感谢Joel Wright提供的倒计时程序。

\begin{flushright}
Graham Hutton\\
诺丁汉, 2006
\end{flushright}
                                           % 序言
%\XeTeXinputencoding "GBK"                                   % 本文件采用GBK编码

\chapter*{译者序}

\begin{quotation}
"A language that doesn't affect the way you think about programming, is not worth knowing".
\begin{flushright}
\textit{-- Alan Perlis(ACM第一任主席,图灵奖得主,1922-1990)}
\end{flushright}
\end{quotation}

《程序员修炼之道》一书作者建议程序员每年应至少学习一门新的语言,以拓宽思维,避免墨守成规。今年我选择了函数式编程语言
Haskell。选择Haskell的理由正如Alan
Perlis所说的那样,Haskell是一门可以影响程序员编程思维的语言,我也期望通过学习Haskell来拓宽我的思维。

开始接触Haskell后,我才发现它在国内是如此的小众(其实在国际上也很小众),国内居然没有正式出版过Haskell相关的中文书籍\footnote{据说Real
world Haskell正在被翻译中},唯一可参考的像样的中文资料就是网上流传的一本免费的由乔海燕翻译的《Yet
Another Haskell Tutorial》
,国内出版的影印版书籍似乎也只有《真实世界的Haskell》(英文名:Real World
Haskell)这一本。

我开始学习Haskell时用的就是那本曾获得过Jolt Award大奖的《Real World
Haskell》影印版,书很厚,是本Haskell大全。但后来发现似乎不太适合初学者。随后又在网上搜索资料,找到了Graham
Hutton 编写的《Programming in
Haskell》\footnote{\url{http://www.cs.nott.ac.uk/~gmh/book.html}}这本教程。与《Real World
Haskell》比起来,《Programming in
Haskell》这本书就显得“单薄”了许多,加起来总共不到200
页。不过这本书却非常适合函数式编程和Haskell的初学者,因为这本书是基于英国诺丁汉大学课程讲义编制而成,经过了多年实际教学检验,并且在这本书的官方主页上还可以下载到与书配套的讲义幻灯片和习题答案。

同样是也是在这本书的主页上,我发现了这本书在2009年就已经出版了日文版和韩文版,这个让我很是受触动,为什么在好书引进方面我们也落后于日韩呢!突然脑中迸发出一个念头:要不我来试试翻译一下这本书,也算是为Haskell在中国的发展做出一些自己的贡献。

于是在Google Code上建立了这个《Programming in Haskell》中文版翻译项目\footnote{\url{http://code.google.com/p/programming-in-haskell-cn}}。

真诚的欢迎大家提出建议和意见,帮助我来审校翻译中存在的问题,共同完成这个项目。

另外这里需声明一点:自己仅是一个Haskell爱好者和初学者,非Haskell牛人。请大家读译稿后谨慎拍砖!

\hspace*{12cm} Tony Bai\\
\hspace*{12cm} October, 2010
                                 % 译者序
%\XeTeXinputencoding "GBK"                                   % 本文件采用GBK编码

\chapter*{中英文对照表}
xx
                                   % 中英文对照表

\tableofcontents                                            % 生成目录
\setcounter{tocdepth}{3}                                    % 设置目录深度

\chapter{导~言}
在这一章节中,我们为本书的后续部分展开打好了基础。我们从回顾函数的概念开始,然后介绍函数式编程的概念,总结Haskell的主要特点和它的历史,最后通过两个小例子“品尝”一下Haskell的味道。

\section{函数}
在Haskell中,一个\textit{函数}是一个映射,它接受一个或多个参数,并产生唯一一个结果。我们可以通过一个等式来定义函数,等式中包含函数名、参数名以及详细描述如何依据参数计算出最终结果的函数体。

例如,一个函数$double$,接受一个数字$x$作为参数,产生的结果为$x + x$,它可通过如下等式定义:\\
\hspace*{1cm} $double~x = x + x$

当一个函数被应用于实际参数时,其结果可通过将实际参数替换函数体中的参数名的方式获得。这个过程可能会立即产生一个不能被进一步简化的结果,比如一个数字。而更为常见的情况是,这个结果是一个含有其他函数程序的表达式, 我们必须以同样的方式处理这个表达式才能得到最终的结果。 

例如,程序$double~3$将函数$double$应用于数字3的结果可通过如下计算过程得出,每一步计算通过花括号里简短注释解释:

\noindent\hspace*{1cm} $double~3$ \\ 
\hspace*{1cm} = \{applying $double$\} \\
\hspace*{1cm} $3 + 3$\\
\hspace*{1cm} = \{applying +\}\\
\hspace*{1cm} $6$

同样,两次应用$double$的内嵌程序$double~(double~2)$的结果可以通过如下计算过程得出:

\noindent\hspace*{1cm} $double~(double~2)$\\
\hspace*{1cm} = \{applying the inner $double$\}\\
\hspace*{1cm} $double~(2~+~2)$\\
\hspace*{1cm} = \{applying~+~\}\\
\hspace*{1cm} $double~4$\\
\hspace*{1cm} = \{applying $double$ \}\\
\hspace*{1cm} $4+4$\\
\hspace*{1cm} = \{applying~+\}\\
\hspace*{1cm} $8$

另外,同样的结果也可以通过先从外层的函数$double$开始计算获得:

\noindent\hspace*{1cm} double~(double~2)\\
\hspace*{1cm} = \{applying the outer $double$\}\\
\hspace*{1cm} $double~2~+~double~2$\\
\hspace*{1cm} = \{applying the first $double$\}\\
\hspace*{1cm} $(2~+~2)~+~double~2$\\
\hspace*{1cm} = \{applying~the~first~+~\}\\
\hspace*{1cm} $4~+~double~2$\\
\hspace*{1cm} = \{applying $double$\}\\
\hspace*{1cm} $4~+~(2~+~2)$\\
\hspace*{1cm} = \{applying~the~second~+\}\\
\hspace*{1cm} $4 + 4$\\
\hspace*{1cm} = \{applying~+~\}\\
\hspace*{1cm} $8$

但是,这个计算过程比我们原来的版本多出两步,因为表达式$double~
2$在第一步中被复制了一份并因此被化简了两次。一般来说,函数在计算过程中应用的顺序不会影响最终的结果值,但它可能会影响到所需步骤的数量,并可能影响计算过程是否终止的判断。本书第12章针对这些问题作了更为详细的探究。

\section{函数式编程}

什么是函数式编程?见仁见智,很难给出一个确切的定义。但总的来说,函数式编程可以被看作一种\textit{编程风格},这种风格的基本计算方式是将函数应用于实际参数。相应的,一门函数式编程语言就是\textit{支持}和\textit{鼓励}使用函数式风格的计算机编程语言。

为了说明这些概念,让我们考虑一个计算从1到$n$的整数和的任务吧。在当前大多数编程语言中,这个任务通常可以通过使用两个可随时改变的存储值变量实现,一个变量从1变到$n$,另外一个变量用来累加总数。

例如,如果我们使用赋值符号:=来改变一个变量的值,使用关键字\textbf{repeat}和\textbf{until}来反复执行一个指令序列,直到某个条件被满足,然后下面的指令序列会计算出所需的总和:

\noindent\hspace*{1cm} $count := 0$\\
\hspace*{1cm} $total := 0$\\
\hspace*{1cm} \textbf{repeat}\\
\hspace*{1cm} \quad $count := count + 1$\\
\hspace*{1cm} \quad $total := total + count$\\
\hspace*{1cm} \textbf{until}\\
\hspace*{1cm} \quad $count = n$

也就是说,我们首先将counter和total这两个变量初始化为零,然后反复递增counter,并把这个值与变量total相加,直到counter达到$n$,此时计算过程停止。

在上述程序中,计算的基本方法是改变存储的值,在某种意义上说,程序执行就是一系列的赋值操作。例如,$n
= 5$时我们得到如下序列,其中最后赋给变量$total$的值就是所需的总和:

\noindent\hspace*{1cm} $count := 0$\\ 
\hspace*{1cm} $total := 0$\\
\hspace*{1cm} $count := 1$\\
\hspace*{1cm} $total := 1$\\
\hspace*{1cm} $count := 2$\\
\hspace*{1cm} $total := 3$\\
\hspace*{1cm} $count := 3$\\
\hspace*{1cm} $total := 6$\\
\hspace*{1cm} $count := 4$\\
\hspace*{1cm} $total := 10$\\
\hspace*{1cm} $count := 5$\\
\hspace*{1cm} $total := 15$\\

通常,这种以改变存储值为基本计算方式的编程语言被称为\textit{命令式语言},因为用这类语言编写的程序由一系列命令式指令构成,这些指令精确描述了计算过程应该如何进行。

现在让我们考虑使用Haskell来计算从1到$n$的整数和。这通常可使用两个库函数实现,一个是$[~..~]$,用于产生从1到$n$之间的数字列表;另外一个是$sum$,用于针对这个列表求和。 

\noindent\hspace*{1cm} $sum [~1..n~]$\\

在这个程序中,计算的基本方法是将函数应用于参数。在这个意义上,程序的执行过程实际上是一系列的函数应用。比如当$n
= 5$时,我们得到如下序列,最终结果就是我们所需要的总和:

\noindent\hspace*{1cm} $sum [~1..5~]$\\
\hspace*{1cm} = \{ applying $[~..~]$ \}\\
\hspace*{1cm} $sum [1,~2,~3,~4,~5]$\\
\hspace*{1cm} = \{ applying $sum$ \}\\
\hspace*{1cm} $1+2+3+4+5$\\
\hspace*{1cm} = \{ applying + \}\\
\hspace*{1cm} $15$

大多数命令式语言都支持一些使用函数编程的形式,所以Haskell程序$sum
[~1..n~]$可以被转化成这些语言。但是,大多数命令式语言不鼓励使用函数式风格编程。比如,大多数语言不鼓励或禁止函数被存储在类似列表的数据结构中,禁止构建类似上面例子中数字列表那样的中间结构;
禁止接受函数作为参数或将函数作为返回值,禁止根据自己定义自己 。相反,Haskell在如何使用函数上没有这些限制,并且提供了一系列功能特点,使得使用函数进行编程既简单又强大。

\section{Haskell的特点}
作为参考,Haskell的主要功能特点都列在了下面,并伴随有提供进一步细节的章节号。

\begin{itemize}
\item 简明的程序 (第二章和第四章)

由于函数式风格抽象层次高的本质,使用Haskell编写的程序往往比用其他语言更加\textit{简明},正如上一节例子中说明的那样。此外,Haskell的语法设计充分考虑了简明的特点,尤其是拥有较少的关键字,并允许使用缩进来表明程序结构。虽然很难作出客观的比较,但Haskell编写的程序往往比用当前其他语言编写的程序短小2-10倍。

\item 强大的类型系统(第三章和第十章)

大多数现代编程语言都包含某种形式的\textit{类型系统}来检测不兼容错误,如试图将一个数字和一个字符相加。Haskell有一个类型系统,它仅从程序员那里获取很少量的类型信息,但却可以在程序执行之前使用一种被称为类型推断的过程自动检查出大量不兼容的错误。Haskell的类型系统也比大多数现代编程语言更为强大,它允许函数是“多态的”和“重载的”。

\item List Comprehensions(第五章)

在计算中一种最常见的构造和操作数据的方法就是使用列表。为此,Haskell提供列表作为语言的一种基本概念,并连同一个简单但功能强大的\textit{comprehension}符号,使用这些符号可以从已有列表中选择或过滤元素来构建新的列表。comprehension符号的使用使得列表上许多公共函数以一种清晰、简明的方式定义出来,而不需要显式的递归。

\item 递归函数(第六章)

大多数实用程序都包含一些形式的重复或循环。在Haskell中,实现循环的基本机制是使用根据自己定义自己的\textit{递归函数}。许多计算都能用递归函数给出一个简单和自然的定义,特别是使用“模式匹配”和“guards”将不同情况分成不同等式时。

\item 高阶函数(第七章)

Haskell是一门\textit{高阶}函数式编程语言,这意味着在函数定义中你可以自由将函数作为参数和结果返回值。使用高阶函数接受常见的编程模式,例如组合两个函数或定义作为语言自身的函数。
更常见的是在Haskell中高阶函数可以用于定义“领域专用语言”,比如列表处理、解析以及交互式编程。

\item Monadic作用(第八章和第九章)

Haskell中的函数都是纯函数,它们接受所有输入作为参数,将所有输出作为结果返回。但是,许多程序需要某种形式的\textit{副作用},这似乎与纯洁性有冲突,比如程序运行时从键盘读取输入或输出结果到屏幕。Haskell提供了一个不损害函数纯洁性的基于\textit{monad}数学概念的处理副作用的统一框架。

\item 惰性求值(第十二章)

Haskell程序的执行使用了一种叫\textit{惰性求值}的技术,这种技术的基本思想是直到其结果是实际需要的时候,计算才应该被执行。除了避免不必要的计算,惰性求值保证程序适时结束,鼓励以使用中间数据结构的模块式风格进行编程,甚至允许使用拥有无穷元素个数的数据结构,比如一个无穷的数字列表。

\item 程序推导

因为在Haskell中程序是纯函数,所以简单的\textit{等式推导}可用于执行程序,变换程序,证明程序属性,甚至能够从他们的行为规范中直接提取出程序。在结合使用归纳方法对递归函数进行推导时,等式推导尤为强大。
\end{itemize}

\section{历史背景}
Haskell的许多特点并非首创,都是由其他语言首次引入的。为了帮助大家了解Haskell的背景,下面简要总结一下有关Haskell语言的一些主要的历史性的发展: 

\begin{itemize}
\item 20世纪30年代,Alonzo Church发明了lambda演算,一种简单但功能强大的数学函数理论。
\item 20世纪50年代,John McCarthy发明了Lisp(列表处理器),Lisp被公认为是世上第一种函数式编程语言。Lisp许多方面受到了lambda演算的影响,但同时仍然接受变量赋值作为语言的一个核心特征。
\item 20世纪60年代,Peter Landin发明了ISWIN(“If you See What I Mean”),第一种纯函数式编程语言,它主要基于lambda演算,并且没有变量赋值。
\item 20世纪70年代,John Backus发明了FP("Fuctional
Programming"),一种特别强调高阶函数和程序推导思想的语言。
\item 同样也是在20世纪70年代,Robin Milner和其他人一起开发了ML(元语言),第一种现代函数式编程语言,引入了多态类型和类型推断思想。
\item 20世纪70年代和80年代,David Turner 开发出许多惰性的函数式编程语言,最终造就了可获得商业支持的Miranda(意为"令人敬佩的")语言的出现。
\item 1987年,一个国际研究委员会发起开发Haskell语言(以逻辑学家Haskell Curry命名),一个标准的惰性函数式编程语言。
\item 2003年,该委员会公布了Haskell的报告,报告中定义了一个期待已久的Haskell的稳定版本,该版本是该语言设计者们十五年工作的成果。
\end{itemize}

值得注意的是,上面提到的三个研究人员 - McCarthy, Backus和Milner各自获得了等同于计算机领域诺贝尔奖的的ACM图灵奖。

\section{品尝Haskell}
我们在结束本章之前通过两个小例子来品尝一下Haskell编程。首先我们回顾一下本章前面使用的函数$sum$,$sum$用于计算列表中一组数字的和。在Haskell中,这个函数可通过如下两个等式定义:

\noindent\hspace*{1cm} $sum [~] = 0$\\
\hspace*{1cm} $sum (x : xs) = x + sum~xs$

第一个等式定义一个空列表的总和是零,同时第二个等式定义一个非空列表的总和是由列表中的第一个数字和后续数字组成的列表$xs$的总和相加在一起获得的。例如,$sum~[1,~2,~3]$的总和计算过程如下:

\noindent\hspace*{1cm} $sum~[1,~2,~3]$\\
\hspace*{1cm} = \{ applying $sum$ \}\\
\hspace*{1cm} $1 + sum~[2,~3]$\\
\hspace*{1cm} = \{ applying $sum$ \}\\
\hspace*{1cm} $1 + (2 + sum~[3])$\\
\hspace*{1cm} = \{ applying $sum$ \}\\
\hspace*{1cm} $1 + (2 + (3 + sum [~]))$\\
\hspace*{1cm} = \{ applying $sum$ \}\\
\hspace*{1cm} $1 + (2 + (3 + 0))$\\
\hspace*{1cm} = \{ applying + \}\\
\hspace*{1cm} $6$

注意,即使函数$sum$使用自身定义自己而形成了递归,它也不会永远循环下去。尤其是每个$sum$都将列表参数的长度减一,直到列表最终变为空表,递归过程也随之终止。将零作为空表的总和再合适不过,因为加法里零不改变加法结果,也就是说对于任何数字$x$,$0
+ x = x$ 且$x + 0 = x$。

在Haskell中,每个函数都有一个描述参数和返回值的\textit{类型},这个类型会自动从函数的定义中推断出来。例如,函数$sum$有以下类型:

\noindent\hspace*{1cm} $Num ~a \Rightarrow [~a~] \rightarrow a$

这个类型指出对于任何数字类型$a$,$sum$是一个将一组这样的数字列表映射到一个单一数字的函数。Haskell支持许多不同类型的数字,其中包括整数,如123,“浮点数”,如3.14159。因此,$sum$可以应用于一个整数列表,正如在上面的计算过程那样,也可以应用于一个浮点数列表。

类型提供了有关函数本质的有用的信息,但更为重要的是,类型的使用使得许多错误可以在程序被执行之前被自动检查出来。尤其是,对于一个程序中的每个函数都会检查其实际参数类型与函数本身的类型是否兼容。例如,试图将函数$sum$应用于一个字符列表将报错,因为字符不是数字类型。

现在让我们考虑一个关于列表的更为有趣的函数吧,这个函数说明了Haskell其他一些方面的特性。假设我们定义了一个名为$qsort$的函数,它由以下两个等式构成:

\hspace*{1cm} $qsort [~] = [~]$\\
\hspace*{1cm} $qsort (x : xs) = qsort~smaller~++~[x]~++~qsort~larger$\\
\hspace*{3cm}                  where\\
\hspace*{4cm}                     smaller = $[a~|~a \leftarrow xs,~a ≤ x ]$\\
\hspace*{4cm}                     larger = $[b~|~b \leftarrow xs,~b > x ]$

在这个定义里,+是一个连接两个列表的操作符。例如$[1,~2,~3]~++~[4,~5] =
[1,~2,~3,~4,~5]$。
相应的,\textbf{where}是一个关键字,用于引入局部定义。在这个例子中,\textit{smaller}列表由$xs$列表中所有小于等于$x$的元素组成,同时larger列表由$xs$列表中所有大于$x$的元素组成。例如,如果$x
= 3$且$xs = [ 5,~1,~4,~2]$,那么$smaller = [1,~2]$,$larger = [5,~4]$。

$qsort$究竟做了什么?首先我们要明确它对仅有一个元素的列表不起作用,在这种意义上,对于任何$x$,$qsort~[x] = [x]$:

\noindent\hspace*{1cm} $qsort~[~x~]$\\
\hspace*{1cm} = \{applying $qsort$\}\\
\hspace*{1cm} $qsort~[~] ++ [x] ++ qsort~[~]$\\
\hspace*{1cm} = \{applying $qsort$\}\\
\hspace*{1cm} $[~]~++~[x]~++~[~]$\\
\hspace*{1cm} = \{applying ++ \}\\
\hspace*{1cm} $[x]$

相应的,现在我们将\textit{qsort}应用到一个样例列表,使用上面的定义化简计算过程:

\noindent\hspace*{1cm} $qsort~[3,~5,~1,~4,~2]$\\
\hspace*{1cm} = \{applying $qsort$\}\\
\hspace*{1cm} $qsort~[1,~2] ++ [3] ++ qsort~[5,~4]$\\
\hspace*{1cm} = \{applying $qsort$\}\\
\hspace*{1cm} $(qsort~[~]~++~[1]~++~qsort~[2])~++~[3]$\\
\hspace*{1cm} $++~(qsort~[4]~++~[5]~++~qsort~[~])$\\
\hspace*{1cm} = \{applying $qsort$ , above property\}\\
\hspace*{1cm} $([~]~++~[1]~++~[2])~++~[3]~++~([4]~++~[5]~++~[~])$\\
\hspace*{1cm} = \{applying ++\}\\
\hspace*{1cm} $[1,~2]~++~[3]~++~[4,~5]$\\
\hspace*{1cm} = \{applying ++\}\\
\hspace*{1cm} $[1,~2,~3,~4,~5]$

总之,$qsort$将示例列表按照数字顺序进行了排序。更一般地说,这个函数产生了一个任意数字列表的排序版本。$qsort$的第一个等式表明空列表是已经被排序了的,而第二个等式则表明任何非空列表都可以通过将第一个数字插入两个列表之间的方式排序,这两个列表是通过将剩余号码与该数字比较获得的,比这个数字小的号码集构成一个列表,比这个数字大的号码集构成另外一个列表。这种排序方法被称为快速排序,并且是已知的排序方法中最好的方法之一。

上面的$quicksort$的实现是一个很好的体现Haskell功能强大、清晰又简明的例子。此外,函数$qsort$也比预期更加通用,即不仅仅适合排序数字,同样适用于任何具备有序值的类型。更确切的说,类型

\noindent\hspace*{1cm} $qsort::Ord~a \Rightarrow [a ] \rightarrow [a ]$

表明对于任何具备有序值的类型,$qsort$是一个提供在这种值列表间映射的函数。Haskell支持多种有序值类型,包括数字、单个字符比如'a'以及字符串比如"abcde"。因此,$qsort$这个函数也可以用于对一个字符列表或一个字符串列表进行排序。

\section{本章备注}
Haskell报告可以从Haskel主页www.haskell.org免费下载,同时这份报告也已经出版(25)。另外Hudak的调查报告(11)更详尽的记录了关于函数式编程语言发展的历史。 

\section{习题}
\begin{enumerate}
\item Give another possible calculation for the result of $double~(double~2)$.
\item Show that $sum~[ x ] = x$ for any number $x$ .
\item Define a function $product$ that produces the product of a list of
numbers, and show using your definition that $product~[2, 3, 4] = 24$.
\item How should the definition of the function $qsort$ be modified so that it produces a reverse sorted version of a list?
\item What would be the effect of replacing ≤ by < in the definition of
$qsort$ ?\\ Hint: consider the example $qsort~[2, 2, 3, 1, 1]$.
\end{enumerate}

                                      % 第一章 导 言
\XeTeXinputencoding "GBK"                                   % ���ļ�����GBK����

\chapter{��һ��}

�ڱ����У���������ʹ��Haskell�ĵ�һ�����������Ƚ���Hugsϵͳ��Prelude��׼�⣬Ȼ����ͺ���Ӧ�õķ��ţ��������ǵĵ�һ��Haskell�ű����������һЩ���ڽű����﷨������

\section{Hugsϵͳ}
����������ǰһ���������ģ����ǿ����ֹ�ִ��һЩС��Haskell����Ȼ����ʵ���У�����ͨ����Ҫһ�������Զ�ִ�г����ϵͳ�����Ȿ���У�����ʹ��һ������ΪHugs�Ľ���ʽϵͳ����Ҳ��ʹ����㷺��Haskellʵ�֡�

Hugs�Ľ���ʽ�ı���ʹ����dz��ʺϽ�ѧ������ԭ�͡�������������������������Ӧ�õ�Ҫ��Ȼ���������Ҫ���ߵ����ܻ�����Ŀ�ִ�г���һЩHaskell�ı�����Ҳ�ǿ��õġ�������ʹ����㷺����Glasgow Haskell�ı����������������Ҳ��һ������ʽ�汾����Hugs�����з�ʽ���ƣ�and can readily be used in its place for the purposes of this book��

\section{��׼Prelude}

��Hugsϵͳ����ʱ�������ȼ���һ����ΪPrelude.hs�Ŀ��ļ���Ȼ����ʾһ��>��ʾ��������ϵͳ���ڵȴ��û�����Ҫ����ı���ʽ�����磬������ļ�������������Ϥ�IJ��������ĺ��������������Ҫ����������ӣ������ˣ�����ָ�����㣬������ʾ��
\begin{verbatim}
>2 + 3 
5
>2-3
-1
> 2* 3
6
> 7 'div' 2
3
>2 ^3
8
\end{verbatim}

ע�⣬������������������'div'����������һ���ʵ��ķ�������ô������Բ����������Ǹ�������

��������ѧ������ָ������ȳ˷��ͳ������и��ߵ����ȼ�������Ҳ���бȼӷ��ͼ������ߵ����ȼ������磬\verb|2 * 3 ^ 4|��ʾ\verb|2 *��3 ^ 4��|����\verb|2 + 3 * 4|��ʾ\verb|2 +��3 * 4��|�����⣬ָ���������ҽ�ϵģ����������������������������ϵġ����磬\verb|2 ^ 3 ^ 4|��ζ��\verb|2 ^ ��3 ^ 4��|����2 - 3 + 4��ָ���ǣ�2 - 3��+ 4����ʵ��Ӧ��ʱ������������ʽ����ʽʹ�����Ŷ�������������������������ø�Ϊ����� 

���˲��������ĺ����⣬������ļ����ṩ��һЩ���õIJ����б��ĺ�������Haskell�У��б��е�Ԫ���÷��������ϣ����Զ��ŷָ�����õ�һЩ�����б��Ŀ⺯��˵�����¡�

\begin{itemize}
\item ��һ���ǿ��б���ѡ����һ��Ԫ��:
> head [1, 2, 3, 4, 5]
1

\item ��һ���ǿ��б���ɾ����һ��Ԫ��:
> tail [1, 2, 3, 4, 5]
[2, 3, 4, 5]

\item ѡ���б��еĵ�n��Ԫ�أ���0��ʼ����):
> [1, 2, 3, 4, 5] !! 2
3

\item ѡ���б��е�ǰn��Ԫ��:
> take 3 [1, 2, 3, 4, 5]
[1, 2, 3]

\item ���б���ɾ��ǰn��Ԫ��:
> drop 3 [1, 2, 3, 4, 5]
[4, 5]

\item ������ij���:
> length [1, 2, 3, 4, 5]
5

\item ���������б���Ԫ��֮��:
> sum [1, 2, 3, 4, 5]
15

\item ���������б���Ԫ��֮��:
> product [1, 2, 3, 4, 5]
120

\item ���������б�:
> [1, 2, 3] ++ [4, 5]
[1, 2, 3, 4, 5]

\item ��ת�б�:
> reverse [1, 2, 3, 4, 5]
[5, 4, 3, 2, 1]

\end{itemize}

���ijЩ����ֵ����׼prelude�е�һЩ�������ܲ������󡣱�����ͼ������һ�����б���ѡ���һ��Ԫ�ض������һ������
\begin{verbatim}
> 1 ��div �� 0
Error
> head [ ]
Error
\end{verbatim}

ʵ���ϣ������ִ���ʱ��HugsϵͳҲ�����һ����Ϣ���ṩһЩ����ԭ�����Ϣ��

��Ϊ�ο�����¼A�����˱�׼Prelude��õ�һЩ���壬��¼B��ʾ�������Haskell���ţ���\verb|^|��+�����ͺ�ʹ����ͨ���̡�

\section{����Ӧ��}

����ѧ�У�������Ӧ�õ�����ͨ����ʾΪ�����Ž���������������������ֵ��˵ı�ʾ�ܼ򵥣�������ֵһ������һ���İ���д���ɡ����磬����ѧ�У�����ʽ

$f (a, b) + c d$

��Ϊ������fӦ�õ���������a��b�ϣ����������c��d�ij˻���ӡ���Haskell�У�����Ӧ��ʹ�ÿո��ʾ��Ȼ����ֵ�������ʽʹ��\verb|*|��������ʾ�����磬����ı���ʽʹ��Haskell��д���£�

\verb|f a b + c * d|

���⣬����Ӧ��ӵ�б��������������ߵ����ȼ������磬f a + b ��Ϊ(f a) + b���±�������һЩ���ӣ�����һ��˵��������������ѧ��Haskell֮��IJ��졣

\begin{table}[htbp]
\label{tab:threesome}
\centering
\begin{tabular}{ll}
\hline
Mathematics & Haskell  \\
\hline
$f(x)$ & f~x \\
$f(x,y)$ & f~x~y \\
$f(g(x))$ & f~(g~x) \\
$f(x,g(y))$ & f~x~(g~y)\\
$f(x)g(y)$ & f~x~*~g~y\\
\hline
\end{tabular}
\end{table}

ע������ı���ʽf~(g~x)��Haskell����Ȼ��Ҫ���ţ���Ϊf~g~ x�����ᱻ����Ϊ������fӦ�õ���������g��x��Ȼ���䱾��ȴ�ǽ�fӦ�õ�һ�������ϣ��ò������ǽ�����gӦ�õ�����x�ϵĽ����A
similar remark holds for the expression f~x~(g~y).

\section{Haskell�ű�}
���˱�׼Prelude���ṩ�ĺ����⣬��Ҳ���Զ����µĺ��������޷���Hugs��>��ʾ���¶����º�����ֻ���ڽű��ж��塣�ű���һ����һϵ�ж�����ɵ��ı��ļ������չ�����Haskell�ű�ͨ������.hs��Ϊ�ļ���׺��������������������ļ����ǡ�

\subsection{�ҵĵ�һ���ű�}
������һ��Haskell�ű�ʱ��������������һֱ�����Ǻ����õģ�һ���������нű��ı༭��������һ������Hugs���ٸ����ӣ��������������ı��༭���������������Ķ��壬������ű���һ����Ϊtest.hs���ļ��У�

\begin{verbatim}
double x = x + x
quadruple x = double (double x)
\end{verbatim}

��Ӧ�ģ��������DZ��ֱ༭�����ڴ��ڴ�״̬����������һ������������Hugs������ָ��ʹ���������½ű���
\verb|> :load test.hs|

����Prelude.hs��test.hs���������ˣ��������ű��еĺ�������������ʹ���ˡ����磺

\begin{verbatim}
> quadruple 10
40
> take (double 2) [1, 2, 3, 4, 5, 6]
[1, 2, 3, 4]
\end{verbatim}

���ڱ���Hugs��������״̬�����Ƿ��ص��༭�����ڡ����������������Ķ������ӵ��ű��У������±����ļ���

\begin{verbatim}
factorial n = product [1 . . n ]
average ns = sum ns ��div �� length ns
\end{verbatim}

����ͬ�������������壺\verb|average ns = div (sum ns ) (length ns )|,
���ǽ�div�������������м������Ȼ��һ������£��κ����������ĺ���������д�ɽ��������÷�������(`)���Ϻ���������֮�����ʽ��

���ű����޸ĺ�Hugs�����Զ��������ǣ�������ʹ���¶���֮ǰ����ִ��һ��reload���
\begin{verbatim}
> :reload
> factorial 10
3628800
> average [1, 2, 3, 4, 5]
3
\end{verbatim}

��Ϊ�ο����±��ܽ���һЩHugs����õ�����ĺ��塣��ע�⣬ÿ���������ͨ�����ĵ�һ���ַ�������д�����磬:load������дΪ:l������:type���ں����ƪ������ϸ���͡�

\begin{table}[htbp]
\label{tab:threesome}
\centering
\begin{tabular}{ll}
\hline
���� & ����\\
\hline
\textit{:load name} & ���ؽű�\textit{name} \\
\textit{:reload} & ���¼��ص�ǰ�ű� \\
\textit{:edit name} & �༭�ű�\textit{name} \\
\textit{:type expr} & ��ʾ\textit{expr}��������Ϣ \\
\textit{:?} & ��ʾ�������� \\
\textit{:quit} & �˳�Hugs \\
\hline
\end{tabular}
\end{table}

\subsection{��������}
����һ���º���ʱ�������Լ�����������ֱ�����Сд��ĸ��ͷ����������Ը������������ĸ������Сд�ʹ�д�������֣��»��ߺ��������š����磬�������ֶ��ǺϷ��ģ�

\textit{myFun~~~fun1~~~arg~2~~~x��}

�����б��еĹؼ����������ж���������ĺ��壬���Ҳ�����Ϊ���������������ʹ�ã�
\begin{verbatim}
case class data default deriving do else
if import in infix infixl infixr instance
let module newtype of then type where
\end{verbatim}

���չ�������Haskell��list����������ͨ����һ����׺s���������ǿ��ܺ��ж��ֵ�����磬һ�����ֿ��ܱ�����Ϊns��һ������ֵ���б����ܻᱻ����Ϊxs��һ���ַ��б����ܱ�������css��

\subsection{���ֹ���}

��һ���ű��У�ÿ��������뾫ȷ�Ĵ���ͬ���п�ʼ�����ֲ��ֹ���ʹ�����ܹ����ݴ�������ȷ��������顣����ű���

\begin{verbatim}
a = b + c
        where 
            b = 1
            c = 2
d = a * 2
\end{verbatim}

ͨ���������Ժ�����Ŀ���b��c����a��������ʹ�õľֲ����塣�����Ҫ��������������ʽ��ͨ�������Ž�һϵ�ж��������������Ҷ���֮������÷ֺŸ��������磬����Ľű�Ҳ����д�ɣ�

\begin{verbatim}
a = b + c
        where 
            {b = 1
            c = 2}
d = a * 2
\end{verbatim}

��һ����˵���������ֹ�����ȷ����������ʹ����ʽ�﷨����������

\subsection{ע��}

�����µĶ��壬�ű�Ҳ�����ע�ͣ���ע�ͽ���Hugs���ԡ�Haskell�ṩ���������͵�ע�ͣ��ֱ��Ϊ��ͨע�ͺ�Ƕ��ע�͡���ͨע���Է���-��ʼ���������쵽��ǰ�еĽ�β���������������ʾ��

\begin{verbatim}
�� Factorial of a positive integer:
factorial n = product [1 . . n ]
�� Average of a list of integers:
average ns = sum ns ��div �� length ns
\end{verbatim}

Ƕ��ע�͵Ŀ�ʼ�ͽ�������Ϊ{-��-}��Ƕ��ע�Ϳ��Կ���У����Ƕ��ע�ͻ����ܰ�������ע�͡�Ƕ��ע������ʱɾ���ű��е�ij�ζ���ʱ�ر����ã�����������ӣ�

\begin{verbatim}
{-
double x
quadruple x
-}
\end{verbatim}

\section{���±�ע}
Hugsϵͳ�ɴ�Haskell��ҳwww.haskell.org���������أ�����Haskell��ҳ�ϻ��ṩ���������õ���Դ��

\section{ϰ��}

\begin{enumerate}
\item ��������ʽ���������������ʽ�Ľ�������
\begin{verbatim}
2 ^ 3 * 4
2 * 3 + 4 * 5 
2 + 3 * 4 ^ 5
\end{verbatim}

\item ʹ��Hugsִ��һ�鱾�����ṩ������

\item
����Ľű��а��������﷨���󣬾�����Щ����ʹ��Hugsȷ����Ľű���������������
\begin{verbatim}
N = a ��div�� length xs
       where
          a = 10
        xs = [1, 2, 3, 4, 5]
\end{verbatim}

\item Show how the library function last that selects the last element of a
non-empty list could be defined in terms of the library functions introduced
in this chapter. Can you think of another possible definition?

\item Show how the library function init that removes the last element from a
non-empty list could similarly be defined in two different ways.
\end{enumerate}
                                       % 第二章 第一步
\chapter{类型和类}
在这一章节中,我们将介绍Haskell中两个最基础的概念:类型和类。我们首先解释什么是类型以及在Haskell中如何使用它,然后介绍一些基本类型以及使用这些基本类型构造更大的类型的方法,详细讨论函数类型,最后介绍一下多态类型和类型类。

\section{基本概念}
\textit{类型}是一组相关值的集合。例如,类型$Bool$包含两个逻辑值$False$和$True$,而类型$Bool
\rightarrow Bool$则包含了所有将$Bool$类型参数映射为$Bool$类型结果的函数,如逻辑非函数$not$。我们使用符号$v::T$来表示\textit{v}是类型\textit{T}的一个值,并可说成\textit{v}具有类型\textit{T}。例如:

\noindent\hspace*{1cm} $False :: Bool$\\
\hspace*{1cm} $True :: Bool$\\
\hspace*{1cm} $not :: Bool \rightarrow Bool$

更一般的是,符号::也可以用于尚未被求值的表达式,这种情况下,$e::T$意为对表达式$e$求值将产生一个类型$T$的值。例如:

\noindent\hspace*{1cm} $not~False :: Bool$\\
\hspace*{1cm} $not~True :: Bool$\\
\hspace*{1cm} $not~(not~False):: Bool$

在Haskell中,每个表达式必须有一个类型,该类型通过一个先于表达式求值的过程计算得到,这个过程被称为\textit{类型推断}。这个过程的关键在于一个函数应用的类型规则。其中规定如果f是一个将类型$A$参数映射为类型$B$结果的函数,$e$是一个$A$类型的表达式,那么$f~~e$具有类型$B$:\\
\\
$\dfrac{f::A \rightarrow B~~e::A}{f~e::B}$
\\
\\
例如,$not~False::Bool$可通过这样的规则推断,该规则使用了这样的事实:$not::Bool
\rightarrow
Bool$和$False::Bool$。另一方面,表达式$not~3$通过上述规则无法推断类型,因为这需要$3::Bool$,但$3$不是一个逻辑值,这是无效的。像$not 3$这样的表达式无法确定类型,也可以说成是包含了一个类型错误,被视为非法表达式。

由于类型推断在求值过程之前,所以Haskell程序是\textit{类型安全}的,也就是说在求值过程中不会发生类型错误。实际上,类型推断能检查出程序中所占错误比例较高的一类错误,它也是Haskell最有用的特性之一。但是注意使用类型推断并不能消除发生在求值阶段的其他类错误的可能性,比如,表达式$1~'div'~0$可以通过类型推断检查,但在求值阶段报错,因为被$0$除的行为是未定义的。

类型安全的不足之处在于一些求值阶段成功的表达式却因类型原因而被拒绝。例如,条件表达式\textbf{if} 
$True$ \textbf{then} $1$ \textbf{else} $False$求值结果为$1$,
但是因包含一个类型错误而被视为无效表达式。
特别是,条件表达式的类型推断规则要求所有可能的结果都具有相同的类型,而在这里例子中,第一种结果为$1$,是一个数字类型,而第二个结果是$False$,是一个逻辑值类型。在实践中,程序员很快就学会了如何在类型系统的限制下工作以及如何避免这些问题。

In Hugs, the type of any expression can be displayed by preceding the expression by the command :type . For example:

在Hugs系统中,任意表达式的类型都可以通过$:type$显示出来,例如:

\noindent\hspace*{1cm} $> :type not$\\
\hspace*{1cm} $not :: Bool \rightarrow Bool$\\
\hspace*{1cm} \\
\hspace*{1cm} $> :type~not~False$\\
\hspace*{1cm} $not~False :: Bool$\\
\hspace*{1cm} \\ 
\hspace*{1cm} $> :type~not~3$\\
\hspace*{1cm} $Error$

\section{基本类型}
\noindent Haskell提供一套内置到语言中的基本类型,下面是对其中最常用的类型描述:
\\
\\
$Bool$~-~ \textbf{逻辑值}\\
这个类型包含了两个逻辑值$False$和$True$。
\\
\\
$Char$~-~\textbf{字符}\\
这个类型包含了普通键盘上提供的所有单个字符,如'$a$','$A$','$3$'和\verb|'_'|,以及拥有特殊效果的控制字符,如\verb|'\n'|(移动到新的一行)和\verb|'\t'|(移动到下一个制表位)。与绝大多数其他编程语言标准一样,单个字符必须用单引号' '括起。
\\
\\
$String$~-~\textbf{字符串}\\
这个类型包含了所有字符序列,诸如"$abc$","$1+2=3$"以及空字符串""。同样与绝大多数其他编程语言标准一样,字符串必须用双引号" "括起。
\\
\\
$Int$~-~\textbf{固定精度整数}\\
这个类型包含诸如$-100$,$0$以及$999$这样的整数,计算机以固定大小的内存存储这些值。例如,Hugs系统中$Int$类型的取值范围在$-231$和$231-1$之间。超出这个范围的将得到非预期的结果。例如,在Hugs系统中对\verb|2^31::Int|(使用::将强制结果为$Int$类型而不是其他的什么数值类型)进行求值将得到一个负值,这是不正确的。
\\
\\
$Integer$~-~\textbf{任意精度整数}\\
该类型包含所有的整数,我们使用足够多的内存储存这个类型的值,从而避免了对该类型值的范围强加上限和下限。例如,使用任何Haskell系统对\verb|2^31::Integer|求值都可以得到正确的结果。

除了对内存和精度的需求不同外,在$Int$和$Integer$之间数字类型的选择还是性能考量之一。特别是,大多数电脑都内置了用来处理固定精度整数的硬件,而任意精度的整数通常必须被看成数字序列,通过速度较慢的软件来处理。
\\
\\
$Float$~-~\textbf{单精度浮点数}\\
这个类型包含带小数点的数字,诸如$-12.34$,$1.0$以及$3.14159$,计算机以固定大小内存存储这些值。浮点一词源于这样一个事实:即小数点后允许的数字位数取决于数的大小。例如使用Hugs对$sqrt ~2 :: Float$求值结果为$1.41421$(库函数$sqrt$用于计算一个数的平方根),其中小数点后有五位数字;而对$sqrt~99999::Float$求值结果为$316.226$,小数点后则只有三位数字。采用浮点数编程是一个专家话题,需要认真对待舍入误差。在这段介绍性的文字中我们对这种类型讲解的很少。

最后,我们注意到一个单个数字可能拥有不止一种数值类型。例如,$3 :: Int$,$3 ::
Integer$和$3 ::
Float$对于数字$3$来说都是有效的类型。这就提出了一个有趣的问题:这些数字在类型推断过程中究竟应该被分配什么类型?这个问题将在本章后面考量类型类时回答。

\section{List类型}
$list$是一个拥有相同类型元素的序列,其元素括在方括号中,并使用逗号分隔。我们将元素类型为T的所有$list$类型写作[$T$]。比如:

\noindent\hspace*{1cm} $[False, True, False ] :: [Bool]$\\
\hspace*{1cm} $[’a’, ’b’, ’c’, ’d’] :: [Char]$\\
\hspace*{1cm} $["One", "Two", "Three"] :: [String]$

在一个list中的元素的个数称为list的\textit{长度}。长度为零的list[]称为空list,而长度为1的lists,如$[False]$和$['a']$,称为singleton lists。注意[[]]和[]是不同的list,前者是singleton list,由唯一一个empty list元素组成,而后者则是一个empty list。

关于list类型需进一步注意三点内容。首先,list类型没有传达其长度信息。例如,$[False,
True]$和$[False, True,
False]$两者都是$[Bool]$类型,即使他们的长度不同。其次,没有关于list中元素类型的限制。目前我们局限在一个我们可以给出的有限范围的例子里,因为到目前为止我们介绍的唯一的非基本类型就是list类型,但是我们可以拥有元素为list的list,诸如:

\noindent\hspace*{1cm} $[[’a’, ’b’], [’c’, ’d’, ’e’]] :: [[Char ]]$

最后,没有任何限制要求一个list必须是有限长度的。特别是,由于在Haskell中使用了惰性求值,具有无限长度的list将正如我们在第12章看到的那样,是自然且实用的。

\section{Tuple类型}
\textit{tuple}是一个可拥有不同类型components的有限长度序列,其components括在圆括号中,并使用逗号分隔。我们用($T_1,T_2,...,T_n$)表示所有tuple类型。对于任意$i$,其取值范围从$1$到$n$,第i个component拥有类型$T_i$。例如:

\noindent\hspace*{1cm} $(False, True) :: (Bool, Bool)$\\
\hspace*{1cm} $(False, ’a’, True) :: (Bool, Char , Bool)$\\
\hspace*{1cm} $("Yes", True, ’a’) :: (String, Bool , Char)$

tuple中components的数量称为$arity$。arity为0的tuple
()称为空tuple。arity为2的tuples称为pairs。arity为3的tuples称为triples,等等。arity为1的tuple,例如($False$),是不允许使用的,因为它们将与显式设置求值顺序的括号的使用相冲突,如在表达式$(1
+ 2) * 3$中。

与list类型一样,关于tuple类型也有三点需要进一步注意的内容。首先tuple类型传达了arity信息。例如,类型($Bool,
Char$)包含了所有由第一个component为$Bool$并且第二个component为$Char$组成的pairs。其次,没有关于tuple中component的类型的限制。例如,我们有以tuple为components的tuple,以list为components的tuple和以tuple为元素的list:

\noindent\hspace*{1cm} $(’a’, (False, ’b’)) :: (Char , (Bool , Char ))$\\
\hspace*{1cm} $([’a’, ’b’], [False, True ]) :: ([Char ], [Bool ])$\\
\hspace*{1cm} $[(’a’, False), (’b’, True)] :: [(Char , Bool )]$

最后,注意tuple必须具有有限长度,以保证tuple类型总是在求值之前被计算得到。

\section{函数类型}
\textit{函数}是由一个类型的参数到另一种类型的结果的映射。我们用$T_1
\rightarrow T_2$表示所有将$T_1$类型参数映射为$T_2$类型结果的函数。例如:

\noindent\hspace*{1cm} $not :: Bool \rightarrow Bool$\\
\hspace*{1cm} $isDigit :: Char \rightarrow Bool$

(库函数$isDigit$判断一个字符是否是一个数字)因为对函数的参数和结果类型没有任何限制,带有一个参数和结果的函数的简单符号已经足以应付多参数和结果的情况了,只需将多个值使用list或tuple打包即可。例如,我们下面定义函数$add$用于计算一整数对的和;定义函数$zeroto$返回一个从$0$到一给定上限值的整数list:

\noindent\hspace*{1cm} $add :: (Int, Int) \rightarrow Int$\\
\hspace*{1cm} $add~(x , y) = x~+~y$\\
\hspace*{1cm} $zeroto :: Int \rightarrow [Int]$\\
\hspace*{1cm} $zeroto~n = [0 . . n ]$

在这些例子中我们遵循了Haskell将函数类型放在函数定义之前作为文档参考的惯例。系统将检查由用户手工提供的类型与类型推断自动计算出的类型两者之间的一致性。

注意没有限制要求函数一定要对它们的参数类型总是有意义。也就是说,对于函数的某些参数来说,其结果是未定义的。比如当list为空时,库函数$head$从一个list中选出第一个元素的行为就是未定义的。

\section{curried函数}
函数可以自由将函数作为结果返回,利用这个事实,拥有多个参数的函数也可以使用另外一种不太明显的方式处理。例如,考虑下面的定义:

\noindent\hspace*{1cm} $add' :: Int \rightarrow (Int \rightarrow Int)$\\
\hspace*{1cm} $add'~x~y = x + y$

类型指出$add'$是一个函数,其参数类型为$Int$,返回结果类型为一个类型为$Int
\rightarrow
Int$的函数。定义本身表明$add'$以整数$x$为参数,后面跟着一个整数$y$,返回结果为$x
+ y$。更确切地说,$add'$以整数$x$为参数,返回一个以整数$y$为参数且返回$x
+ y$的函数。

注意函数$add'$产生的最终结果与上一节中的函数$add$相同。然而函数$add$将其两个参数打包为一个pair一次处理完毕,而函数$add'$则一次仅接受处理一个参数。正如这两个函数的类型所反映的:

\noindent\hspace*{1cm} $add :: (Int, Int) \rightarrow Int$\\
\hspace*{1cm} $add' :: Int \rightarrow (Int \rightarrow Int)$

对于有两个以上参数的函数,也可以使用同样的技术处理,通过返回以函数为返回值的函数,等等。例如,函数$mult$接受三个参数,每次接受一个,并返回它们的乘积,可以定义如下:

\noindent\hspace*{1cm} $mult :: Int \rightarrow (Int \rightarrow (Int
\rightarrow Int))$\\
\hspace*{1cm} $mult~x~y~z = x~*~y~*~z$

这个定义表明$mult$接受整数$x$并返回一个函数,后者依次接受整数$y$并返回另外一个函数,最后这个函数接受整数$z$,并最终返回结果$x~*~y~*~z$。

诸如$add'$和$mult$这样的每次接受一个参数的函数被称为$curried$。除了本身有吸引力之外,curried函数也比接受tuple作为参数的函数更加灵活,因为一些有用的函数通常可以通过\textit{部分应用}到参数不完整的curried函数来实现。例如,一个完成递增功能的函数可以通过只需一个参数的curried函数$add'$的部分应用$add~1
:: Int \rightarrow Int$实现。

为避免在使用curried函数工作时过度使用括号,我们采纳了两个简单的惯例。首先,类型中使用的箭头$\rightarrow$是右结合的,例如:

\noindent\hspace*{1cm} $Int \rightarrow Int \rightarrow Int \rightarrow Int$

意为

\noindent\hspace*{1cm} $Int \rightarrow (Int \rightarrow (Int \rightarrow Int))$

然而使用空格表示的函数应用则是左结合的,例如:

\noindent\hspace*{1cm} $mult~x~y~z$

意为

\noindent\hspace*{1cm} $((mult~x)~y)~z$

除非显式需要tuple,Haskell中的所有接受多个参数的函数一般都会被定义成curried函数,并且使用上面的两个惯例来减少需要使用的括号的数量。

\section{多态类型}
库函数\textit{length}用于计算任意list的长度,无论list中元素是什么类型的。比如,它可以用于计算整型list、字符串型list甚至是函数类型list的长度:\\
\hspace*{1cm} > $length~[1, 3, 5, 7]$\\
\hspace*{1cm} $4$\\
\hspace*{1cm} > $length~["Yes", "No"]$\\
\hspace*{1cm} $2$\\
\hspace*{1cm} > $length~[isDigit , isLower , isUpper]$\\
\hspace*{1cm} $3$

The idea that length can be applied to lists whose elements have any type is
made precise in its type by the inclusion of a \textit{type
variable}。类型变量必须以小写字母开头,通常命名为\textit{a,b,c}等。例如,\textit{length}的类型如下:\\
\hspace*{1cm} $length :: [a] \rightarrow Int$

即对任意类型\textit{a},函数\textit{length}具有类型$[a]
\rightarrow [Int]$。包含一个或多个类型变量的类型被称作多态的(“多种形式")。as
is the expression with such a type。因此$[a] \rightarrow
Int$是一个多态类型,\textit{length}是一个多态函数。更普遍的是,标准Prelude中提供的很多函数都是多态的,例如:\\
\hspace*{1cm} $fst :: (a, b) \rightarrow a$\\
\hspace*{1cm} $head :: [a] \rightarrow a$\\
\hspace*{1cm} $take :: Int \rightarrow [a] \rightarrow [a]$\\
\hspace*{1cm} $zip :: [a] \rightarrow [b] \rightarrow [a, b]$\\
\hspace*{1cm} $id :: a \rightarrow a$


                                 % 第三章 类型与类

\end{document}

%
% } body end
%
