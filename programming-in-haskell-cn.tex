%
% TeX模板
%
% 中文支持方案: XeTeX + xeCJK
% author: Tony Bai
% 
% compile: 
%     xelatex programming-in-haskell-cn.tex
%

\documentclass[a4paper,11pt,titlepage]{book}                % 五号字

%
% preamble begin {
%

\usepackage{fontspec}
\usepackage{xunicode} 
\usepackage{xltxtra}                                        % xltxtra include the cmd \XeTeX

\XeTeXlinebreaklocale "zh"
\XeTeXlinebreakskip = 0pt plus 1pt minus 0.1pt

\usepackage{titletoc}                                       % 控制目录
\usepackage{appendix}                                       % 控制附录

\usepackage[colorlinks,
            linkcolor=black,
            citecolor=black]{hyperref}                      % \url

% 页边距设置
\usepackage[top=1.2in,bottom=1.2in,left=1.2in,right=1in]{geometry}

% 封面设置
\title{Haskell程序设计}
\author{著:Graham Hutton\\
        译:Tony Bai\footnote{\url{http://bigwhite.blogbus.com}}}
\date{October, 2010}

% 页眉页脚设置
\usepackage{fancyhdr}
\pagestyle{fancy}
\fancyhf{}                                                  % 清空页眉页脚
\fancyhead[LE,RO]{\thepage}                                 % 偶数页左,奇数页右
\fancyhead[RE]{\leftmark}                                   % 偶数页右
\fancyhead[LO]{\rightmark}                                  % 奇数页左
\fancypagestyle{plain}{
\fancyhf{}                                                  % 重定义plain页面样式
\renewcommand{\headrulewidth}{0pt}
}
\renewcommand\chaptermark[1]{\markboth{\chaptername~ #1}{}}
\renewcommand\sectionmark[1]{\markright{\thesection~ #1}}

% 行距
\renewcommand{\baselinestretch}{1.25}

% 章节设置
\usepackage{titlesec} 
\titleformat{\chapter}{\centering\huge}{第\thechapter{}章}{1em}{\textbf}

% xeCJK设置
\usepackage[slantfont, boldfont, CJKaddspaces]{xeCJK}
\setmainfont{Times New Roman}                                % view the font list through the cmd "fc-list :lang=en"
\setCJKmainfont{WenQuanYi Micro Hei}                         % view the font list through the cmd "fc-list :lang=zh"
\setCJKfamilyfont{song}{WenQuanYi Micro Hei}
\setsansfont{WenQuanYi Micro Hei}

% 重定义设置
\renewcommand{\chaptername}{第{\thechapter}章}
\renewcommand{\contentsname}{目~录}
\renewcommand{\indexname}{索引}
\renewcommand{\listfigurename}{插图目录}
\renewcommand{\listtablename}{表格目录}
\renewcommand{\figurename}{图}
\renewcommand{\tablename}{表}
\renewcommand{\appendixname}{附录}
\renewcommand{\appendixpagename}{附录}
\renewcommand{\appendixtocname}{附录}
%\renewcommand\refname{参考文献}

\usepackage[fleqn]{amsmath}                                 % 

%
% } preamble end
%


%
% body begin {
%

\begin{document}

\maketitle                                                  % 生成title

\chapter*{Haskell程序设计}
\textit{待译}

                                        % half_title
\chapter*{版权声明}
由于尚未获得原作者授权,所以本项目仅用于学习和交流之用,不得用于任何商业用途。
                                         % 版权声明
\chapter*{序}
\addcontentsline{toc}{chapter}{序}
\begin{quotation}
...这个世界上有两种设计软件的方法:一种是使设计尽量的简化,以至于明显没有任何缺陷;而另一种是使设计尽量的复杂,以至于找不到明显的缺陷。第一种方法更加困难。
\begin{flushright}
\textit{ -- Tony Hoare,1980 ACM图灵奖演讲}
\end{flushright}
\end{quotation}
这本书讲述了一种以简单、清晰和优雅为关键目标的编程方法。更具体的说,它是一本使用Haskell语言介绍这种函数式编程方法的入门书籍。

函数式编程与当前大多数编程语言比如Java,C++,C和Visual
Basic所提倡的风格相差很大。特别是当前大多数语言是与底层硬件紧密关联的,在这种意义上,编程的基本思想就是修改存储的值。与此相反,Haskell则提倡一种更为抽象的编程风格,这种风格的基本思想则是将函数应用于参数。正如我们将要看到的,站上更高的层次可以让我们拥有更为简单的程序,并且支持一系列强大的构造和推导程序的新方法。

本书主要面对具备大专水平的学习计算机科学的学生,但也同样广泛适用于那些想了解和学习Haskell编程的读者。你不需要拥有任何编程经验,所有的概念都从基本原理讲起,并伴以精心挑选的例子。

本书使用的Haskell版本是Haskell
98,Haskell语言的标准版本,Haskell设计者花费了15年才最终发布了这个标准。由于这里仅仅是入门引导,所以我们不会去涉及
Haskell语言以及其相关库的所有方面。本书大约有一半内容是专门介绍该语言的主要特点的,另一半则包括了Haskell编程的例子和案例研究。每个章节还包括了一系列的习题以及关于进一步阅读更高级、更专业主题的建议。

本书基于课程材料编写而成,这些课程材料在诺丁汉大学经过了多年的改进和课程测试。通过20学时的授课以及大约40小时的自学、实验室实践和编程作业,你就可以学完本书的大部分内容。然而,你还需要更多的时间详细学习一下后面的一些章节以及一些编程例子。

本书的官方网站提供了一系列辅助资料,包括每个章节的幻灯片和一些扩展例子的Haskell代码。教师还可以通过发送电邮到solutions@cambridge.org得到每个章节练习题的标准答案以及大量带有标准答案的试题集。

\begin{flushleft}
致谢\\
\end{flushleft}

诺丁汉大学编程社团为函数式编程的研究与教学提供了一个极好的环境。感谢学校提供假期让我可以进行本书的写作。感谢所有学生和讲师们关于我的Haskell课程的反馈;
感谢Thorsten Altenkirch, Neil Ghani, Mark Jones (现在在波特兰州), Conor McBrideh和Henrik Nilsson在FOP社区与我进行的关于函数式编程的想法以及如何表达这些想法的交流和讨论。

我还要感谢David Tranah和Dawn Preston在剑桥大学出版社出色的编辑工作; 感谢Mark
Jones的Haskell解释器; 感谢Ralf Hinze和Andres Loh提供的lhs2TeX排版系统;感谢Rik
van Geldrop和Jaap van der Woude关于使用本书草稿的反馈; 感谢Kees van den Broek,
Frank Heitmann和 Bill Tonkin指出的本书的错误;感谢Ian
Bayley和所有匿名评审者极富价值的评论;感谢Joel Wright提供的倒计时程序。

\begin{flushright}
Graham Hutton\\
诺丁汉, 2006
\end{flushright}
                                           % 序言
\XeTeXinputencoding "GBK"                                   % ���ļ�����GBK����

\chapter*{������}
xx
                                 % 译者序
\chapter*{中英文对照表}
\textit{待补充}
                                   % 中英文对照表

\tableofcontents                                            % 生成目录
\setcounter{tocdepth}{3}                                    % 设置目录深度

\XeTeXinputencoding "GBK"                                   % ���ļ�����GBK����

\chapter{��~��}
����һ�½��У�����Ϊ����ĺ��������趨�˽׶Ρ����Ǵӻع˺����ĸ��ʼ��Ȼ����ܺ���ʽ��̵ĸ���ܽ�Haskell����Ҫ�ص��������ʷ�����ͨ������С���ӡ�Ʒ����һ��Haskell��ζ����

\section{����}
��Haskell�У�һ��������һ��ӳ�䣬����һ������������������Ψһһ����������ǿ���ͨ��һ����ʽ�����庯������ʽ�а������������֡����������������Լ���ϸ����������ݲ������������ĺ����塣

���磬һ����Ϊdouble�ĺ�����������һ������x�������Ľ��Ϊx~+~x����ͨ�����µ�ʽ���壺\\
\hspace*{1cm} \textit{double x = x + x}

��һ��������Ӧ�õ�ʵ�ʲ���ʱ��������ͨ����ʵ�ʲ����滻�������еIJ������Ƶķ�ʽ��á�������̿��ܻ���������һ�����ܱ���һ���򻯵Ľ��������һ�����֡�����Ϊ����������ǣ���������һ������������������ı���ʽ�������뱻��ͬ���ķ�ʽ�������ܲ������յĽ����

���磬����\textit{double} 3������\textit{double}Ӧ�õ�����3�Ľ����ͨ�����¼�����̵ó���ÿһ������ͨ������������ע�ͽ��ͣ�\\
\hspace*{1cm} \textit{double} 3\\ 
\hspace*{1cm} = \{applying \textit{double}\} \\
\hspace*{1cm} 3 + 3\\
\hspace*{1cm} = \{applying +\}\\
\hspace*{1cm} 6 

ͬ��������Ӧ��double����Ƕ����double (double 2)�Ľ������ͨ�����¼�����̵ó���

\[ \begin{split}
double~(double~2)\\
= \{applying~the~inner~double\}\\
double~(2~+~2)\\
= \{applying~+~\}\\
double~4\\
= \{applying~double \}\\
4+4\\
= \{applying~+\}\\
8
\end{split} \]

���⣬ͬ���Ľ��Ҳ����ͨ���ȴ����ĺ���double��ʼ�����ã�
\[ \begin{split}
double~(double~2)\\
=\{applying~the~outer~double\}\\
double~2~+~double~2\\
=\{applying~the~first~double\}\\
(2~+~2)~+~double~2\\
=\{applying~the~first~+~\}\\
4~+~double~2\\
= \{applying~double\}\\
4~+~(2~+~2)\\
= \{applying~the~second~+\}\\
4+4\\
= \{applying~+~\}\\
8
\end{split} \]

���ǣ����������̱�����ԭ���İ汾�����������Ϊ����ʽdouble 2�ڵ�һ���б�������һ�ݲ���˱����������Ρ�һ����˵�������ڼ��������Ӧ�õ�˳�򲻻�Ӱ�����յĽ��ֵ���������ܻ�Ӱ�쵽���貽���������������Ӱ���������Ƿ���ֹ���жϡ������12�������Щ�������˸�Ϊ��ϸ��̽����

\section{����ʽ���}

ʲô�Ǻ���ʽ��̣����ʼ��ǣ����Ѹ���һ��ȷ�еĶ��塣���ܵ���˵������ʽ��̿��Ա�����һ�ֱ�̷�����ַ��Ļ������㷽ʽ�ǽ�����Ӧ����ʵ�ʲ�������Ӧ�ģ�һ�ź���ʽ������Ծ���֧�ֺ͹���ʹ�ú���ʽ���ļ����������ԡ�

Ϊ��˵����Щ��������ǿ���һ�������1��n�������͵�����ɡ��ڵ�ǰ�������������У��������ͨ������ͨ��ʹ����������ʱ�ı�Ĵ洢ֵ����ʵ�֣�һ��������1�䵽n������һ�����������ۼ�������

���磬�������ʹ�ø�ֵ����:=���ı�һ��������ֵ��ʹ�ùؼ���repeat��until������ִ��һ��ָ�����У�ֱ��ij�����������㣬��ô�����ָ�����м����������ܺͣ�

\begin{verbatim}
count := 0
total := 0
repeat
    count := count + 1
    total := total + count
until
    count = n
\end{verbatim}

Ҳ����˵���������Ƚ����������ܺ�������������ʼ��Ϊ�㣬Ȼ�󷴸��������������������ֵ���ܺͱ�����ӣ�ֱ���������ﵽn����ʱ�������ֹͣ��

�����������У�����Ļ��������Ǹı�洢��ֵ����ij��������˵������ִ�о���һϵ�еĸ�ֵ���������磬n
= 5ʱ���ǵõ��������У�������󸳸�����total��ֵ����������ܺͣ�

\begin{verbatim}
count := 0
total := 0
count := 1
total := 1
count := 2
total := 3
count := 3
total := 6
count := 4
total := 10
count := 5
total := 15
\end{verbatim}

ͨ���������Ըı�洢ֵΪ�������㷽ʽ�ı�����Ա���Ϊ����ʽ���ԣ���Ϊ���������Ա�д�ij�����һϵ������ʽָ��ɣ���Щָ�ȷ�����˼������Ӧ����ν��С�

���������ǿ���ʹ��Haskell�������1��n�������͡���ͨ��ʹ�������⺯����һ����[..]�����ڲ�����1��n֮��������б�������һ����sum�������������б���͡� 

\begin{verbatim}
sum [1..n]
\end{verbatim}

����������У�����Ļ��������ǽ�����Ӧ���ڲ���������������ϣ������ִ�й���ʵ������һϵ�еĺ���Ӧ�á����統n = 5ʱ�����ǵõ��������У����ս��������������Ҫ���ܺͣ�

\begin{verbatim}
sum [1..5]
= { applying [..] }
sum [1, 2, 3, 4, 5]
= { applying sum }
1+2+3+4+5
= { applying + }
15
\end{verbatim}

���������ʽ���Զ�֧��һЩʹ�ú�����̵���ʽ������Haskell����sum [1..n]���Ա�ת������Щ���ԡ����ǣ����������ʽ���Բ�����ʹ�ú���ʽ����̡����磬��������Բ��������ֹ�������洢�������б������ݽṹ�У��򹹽��������������������б��������м�ṹ; ����պ�����Ϊ�����򽫺�����Ϊ����ֵ; ���Լ����塣�෴��Haskell�����ʹ�ú�����û����Щ���ƣ������ṩ��һϵ�й�����ɫ��ʹ��ʹ�ú������б�̼ȼ���ǿ��

\section{Haskell���ص�}
\begin{itemize}
\item �����ij��� (�ڶ��º͵�����)

���ں���ʽ�������θߵı��ʣ�ʹ��Haskell��д�ij������������������Ը��Ӽ�����������һ��������˵�������������⣬Haskell���﷨��Ƴ�ֿ����˼������ص㣬������ӵ�н��ٵĹؼ��֣�������ʹ����������ʾ����ṹ����Ȼ���������͹۵ıȽϣ���Haskell��д�ij���������������ǰ���Ա�д�ij����С2-10����\newline

\item ǿ�������ϵͳ�������º͵�ʮ�£�

������ִ�������Զ�����ij����ʽ������ϵͳ����ⲻ���ݴ�������ͼ��һ�����ֺ�һ���ַ���ӡ�Haskell��һ������ϵͳ�������ӳ���Ա�����ȡ��������������Ϣ����ȴ�����ڳ���ִ��֮ǰʹ��һ�ֱ���Ϊ�����ƶϵĹ����Զ��������������ݵĴ���Haskell������ϵͳҲ�ȴ�����ִ�������Ը�Ϊǿ�������������ǡ���̬�ġ��͡����صġ���\newline

\item �б����⣨�����£�

�ڼ�����һ������Ľṹ���Ͳ������ݵķ�������ʹ���б���Ϊ�ˣ�Haskell�ṩ�б���Ϊ���Ե�һ�ֻ����������ͬһ���򵥵�����ǿ���������ţ�ʹ����Щ�������ǿ���ͨ���������б���ѡ������Ԫ�����������б���������ŵ�ʹ��ʹ���б������๫��������һ�������������ķ�ʽ���������������Ҫ��ʽ�ĵݹ顣

\item �ݹ麯���������£�

�����ʵ�ó��򶼰���һЩ��ʽ���ظ���ѭ������Haskell�У�ʵ��ѭ���Ļ���������ʹ��Ƕ�����Լ�����ĵݹ麯����������㶼���õݹ麯������һ���򵥺���Ȼ�Ķ��壬particularly when ��pattern matching�� and ��guards�� are used to separate different cases into
different equations��

\item �߽ף������£�

Haskell��һ�Ÿ߽׺���ʽ������ԣ�����ζ���ں�����������������ɽ�������Ϊ�����ͽ������ֵ��ʹ�ø߽׺������ܳ����ı��ģʽ��such as composing two functions, to be defined as functions within the language itself.
������������Haskell�и߽׺����������ڶ��塰�����ض����ԡ��������б������������Լ�����ʽ��̡�

\item Monadic���ã��ڰ��º͵ھ��£�

Haskell�еĺ������Ǵ����������ǽ�����������Ϊ�����������������Ϊ������ء����ǣ����������Ҫij����ʽ�ĸ����ã����ƺ��봿�����г�ͻ�����統��������ʱ�Ӽ��̶�ȡ���������������Ļ��Haskell�ṩ��һ�����𺦺��������ԵĻ���monad��ѧ����Ĵ��������õ�ͳһ��ܡ�

\item ������ֵ����ʮ���£�

Haskell�����ִ��ʹ����һ�ֽж�����ֵ�ļ��������ּ����Ļ���˼����ֱ��������ʵ����Ҫ��ʱ�򣬼����Ӧ�ñ�ִ�С����˱��ⲻ��Ҫ�ļ��㣬������ֵ��֤������ʱ������������ʹ���м����ݽṹ��ģ��ʽ�����б�̣���������ʹ��ӵ������Ԫ�ظ��������ݽṹ������һ������������б���

\item ��������

��Ϊ��Haskell�г����Ǵ����������Լ򵥵ĵ�ʽ����������ִ�г��򣬱任����֤���������ԣ������ܹ������ǵ���Ϊ�淶��ֱ����ȡ�������ڽ��ʹ�ù��ɷ����Եݹ麯����������ʱ����ʽ�����ر�ǿ��
\end{itemize}

\section{��ʷ����}
Haskell��������ɫ�����״������������������״�����ġ�To help place Haskell in context�������Ҫ�ܽ�һ���й�Haskell���Ե�һЩ��Ҫ����ʷ�Եķ�չ�� 

\begin{itemize}
\item 20����30�����Alonzo Church������lambda���㣬һ�ּ򵥵�����ǿ�����ѧ�������ۡ�
\item 20����50�����John McCarthy������Lisp(�б�������)��Lisp������Ϊ�����ϵ�һ�ֺ���ʽ������ԡ�Lisp���෽���ܵ���lambda�����Ӱ�죬��ͬʱ��Ȼ���ܱ�����ֵ��Ϊ���Ե�һ������������
\item 20����60�����Peter Landin������ISWIN(��If you See What I Mean��)����һ�ִ�����ʽ������ԣ�����Ҫ����lambda���㣬����û�б�����ֵ��
\item 20����70�����John Backus������FP("Fuctional Programming")��һ���ر�ǿ���߽׺����ͳ�������˼������ԡ�
\item ͬ��Ҳ����20����70�����Robin Milner��������һ�𿪷���ML(Ԫ���ԣ�����һ���ִ�����ʽ������ԣ������˶�̬���ͺ������ƶ�˼�롣
\item 20����70�����80�����David Turner ������������Եĺ���ʽ������ԣ���������˿ɻ����ҵ֧�ֵ�Miranda����Ϊ"���˾����"�����Եij��֡�
\item 1987�꣬һ�������о�ίԱ�ᷢ�𿪷�Haskell���ԣ����߼�ѧ��Haskell Curry��������һ����׼�Ķ��Ժ���ʽ������ԡ�
\item 2003�꣬��ίԱ�ṫ����Haskell�ı��棬�����ж�����һ���ڴ��Ѿõ�Haskell���ȶ��汾���ð汾�Ǹ������������ʮ���깤���ijɹ���
\end{itemize}

ֵ��ע����ǣ������ᵽ�������о���Ա - McCarthy, Backus��Milner���Ի���˵�ͬ�ڼ��������ŵ�������ĵ�ACMͼ�齱��

\section{Ʒ��Haskell}
�����ڽ�������֮ǰͨ������С������Ʒ��һ��Haskell��̡��������ǻع�һ�±���ǰ��ʹ�õĺ���sum��sum���ڼ����б���һ�����ֵĺ͡���Haskell�У����������ͨ������������ʽ���壺

\begin{verbatim}
sum [] = 0
sum (x : xs) = x + sum xs
\end{verbatim}

��һ����ʽ����һ�����б����ܺ����㣬ͬʱ�ڶ�����ʽ����һ���ǿ��б����ܺ������б��еĵ�һ�����ֺͺ���������ɵ��б�xs���ܺ������һ���õġ����磬sum [1��2��3]���ܺͼ���������£�

\begin{verbatim}
sum [1, 2, 3]
= { applying sum }
1 + sum [2, 3]
= { applying sum }
1 + (2 + sum [3])
= { applying sum }
1 + (2 + (3 + sum [ ]))
= { applying sum }
1 + (2 + (3 + 0))
= { applying + }
6
\end{verbatim}

ע�⣬��ʹ����sumʹ�����������Լ����γ��˵ݹ飬��Ҳ������Զѭ����ȥ��������ÿ��sum�����б������ij��ȼ�һ��ֱ���б����ձ�Ϊ�ձ����ݹ����Ҳ��֮��ֹ��������Ϊ�ձ����ܺ��ٺ��ʲ�������Ϊ�ӷ����㲻�ı�ӷ������Ҳ����˵�����κ�����x��0 + x =x��x + 0 = x��

��Haskell�У�ÿ����������һ�����������ͷ���ֵ�����ͣ�������ͻ��Զ��Ӻ����Ķ������ƶϳ��������磬����sum���������ͣ�
\begin{verbatim}
sum a => [a] -> a
\end{verbatim}

�������ָ�������κ���������a��sum��һ����һ�������������б�ӳ�䵽һ����һ���ֵĺ�����Haskell֧�����಻ͬ���͵����֣����а�����������123����������������3.14159����ˣ�sum����Ӧ����һ�������б�������������ļ������������Ҳ����Ӧ����һ���������б���

�����ṩ���йغ������ʵ����õ���Ϣ������Ϊ��Ҫ���ǣ����͵�ʹ��ʹ�������������ڳ���ִ��֮ǰ���Զ��������������ǣ�����һ�������е�ÿ��������������ʵ�ʲ��������뺯�������������Ƿ���ݡ����磬��ͼ������sumӦ����һ���ַ��б�����b��������Ϊ�ַ������������͡�

���������ǿ���һ�������б��ĸ�Ϊ��Ȥ�ĺ����ɣ��������˵����Haskell����һЩ��������ԡ��������Ƕ�����һ����Ϊqsort�ĺ�������������������ʽ���ɣ�

\begin{verbatim}
qsort [ ] = []
qsort (x : xs) = qsort smaller ++ [x ] ++ qsort larger
                 where
                    smaller = [a | a <- xs, a �� x ]
                    larger = [b | b <- xs, b > x ]
\end{verbatim}

����������+��һ�����������б��IJ�����������[1, 2, 3] ++ [4, 5] = [1, 2, 3, 4, 5].  ��Ӧ�ģ�where��һ���ؼ��֣���������ֲ����塣����������У�smaller�б���xs�б�������С�ڵ���x��Ԫ����ɣ�ͬʱlarger�б��� xs�б������д��ڵ���x��Ԫ����ɡ����磬���x = 3��xs = [ 5, 1, 4, 2]����ôsmaller = [1, 2]��larger = [5, 4]��

qsort��������ʲô����������Ҫ��ȷ���Խ���һ��Ԫ�ص��б��������ã������������ϣ������κ�x��qsort [x] =[x]:

\begin{verbatim}
qsort [ x ]
= { applying qsort }
qsort [ ] + [ x ] + qsort [ ]
= { applying qsort }
[ ] + [x ] + [ ]
= { applying + }
[x ]
\end{verbatim}

��Ӧ�ģ��������ǽ�qsortӦ�õ�һ�������б���ʹ������Ķ��廯�������̣�

\begin{verbatim}
qsort [3, 5, 1, 4, 2]
= { applying qsort }
qsort [1, 2] ++ [3] ++ qsort [5, 4]
= { applying qsort }
(qsort [ ] ++ [1] ++ qsort [2]) ++ [3]
++ (qsort [4] ++ [5] ++ qsort [ ])
= { applying qsort , above property }
([ ] ++ [1] ++ [2]) ++ [3] ++ ([4] ++ [5] ++ [ ])
= { applying ++ }
[1, 2] ++ [3] ++ [4, 5]
= { applying ++ }
[1, 2, 3, 4, 5]
\end{verbatim}

��֮��qsort��ʾ���б���������˳����������򡣸�һ���˵���������������һ�����������б�������汾��qsort�ĵ�һ����ʽ�������б����Ѿ��������˵ģ����ڶ�����ʽ������κηǿ��б�������ͨ������һ�����ֲ��������б�֮��ķ�ʽ�����������б���ͨ����ʣ�����������ֱȽϻ�õģ����������С�ĺ��뼯����һ���б�����������ִ�ĺ��뼯��������һ���б����������򷽷�����Ϊ�������򣬲�������֪�����򷽷�����õķ���֮һ��

�����quicksort��ʵ����һ���ܺõ�����Haskell����ǿ�������ּ��������ӡ����⣬����qsortҲ��Ԥ�ڸ���ͨ�ã����������ʺ��������֣�ͬ���������κξ߱�����ֵ�����͡���ȷ�е�˵������

\begin{verbatim}
qsort::Ord a => [a ] -> [a ]
\end{verbatim}

���������κξ߱�����ֵ�����ͣ�qsort��һ���ṩ������ֵ�б���ӳ��ĺ�����Haskell֧�ֶ�������ֵ���ͣ��������֡������ַ�����'a'�Լ��ַ�������"abcde"����ˣ�qsort�������Ҳ�������ڶ�һ���ַ��б���һ���ַ����б���������

\section{���±�ע}
Haskell������Դ�Haskel��ҳwww.haskell.org������أ�ͬʱ��ݱ���Ҳ�Ѿ�����(25)������Hudak�ĵ��鱨��(11)���꾡�ļ�¼�˹��ں���ʽ������Է�չ����ʷ�� 

\section{ϰ��}
\begin{enumerate}
\item Give another possible calculation for the result of double (double 2).
\item Show that sum [ x ] = x for any number x .
\item Define a function product that produces the product of a list of numbers, and show using your definition that product [2, 3, 4] = 24.
\item How should the definition of the function qsort be modified so that it
produces a reverse sorted version of a list?
\item What would be the effect of replacing �� by < in the definition of qsort ?\\ Hint: consider the example qsort [2, 2, 3, 1, 1].
\end{enumerate}

                                      % 第一章 导 言
\chapter{第一步}

在本章中,我们将迈出使用Haskell的第一步。我们首先介绍Hugs系统和Prelude标准库,然后解释函数应用的记法,开发我们的第一个Haskell脚本,最后讨论一些关于脚本的语法惯例。

\section{Hugs系统}
正如我们在前一章所看到的,我们可以手工执行一些小的Haskell程序,然而在实践中,我们通常需要一个可以自动执行程序的系统。在这本书中,我们使用一个被称为\textit{Hugs}的交互式系统,它也是使用最广泛的Haskell实现。

Hugs的交互式的本质使得其非常适合教学和制作原型,并且它的性能可以满足绝大多数应用的要求。然而,如果需要更高的性能或独立的可执行程序,也有一些Haskell的编译器是可供选择的,这其中使用最广泛的是Glasgow
Haskell的编译器,这个编译器也有一个和Hugs的运行方式类似的交互式版本,并且该版本可以很容易的在本书中使用。

\section{标准Prelude}

当Hugs系统启动时,它首先加载一个名为\textit{Prelude.hs}的库文件,然后显示一个>提示符,表明系统正在等待用户输入待求值的表达式。例如,这个库文件定义了许多熟知的操作整数的函数,包括加,减,乘,除和求幂五个主要算术运算,如下所示:

\noindent\hspace*{1cm} $>2~+~3$\\
\hspace*{1cm} $5$\\
\hspace*{1cm} $> 2~-~3$\\
\hspace*{1cm} $-1$\\
\hspace*{1cm} $> 2~*~3$\\
\hspace*{1cm} $6$\\
\hspace*{1cm} $> 7~`div`~2$\\
\hspace*{1cm} $3$\\
\hspace*{1cm} $> 2$ \verb|^| $3$\\
\hspace*{1cm} $8$

注意,整数除法操作符记作$`div`$,如果结果是一个真分数,那么将向下圆整到最近的那个整数。

按常规数学惯例,求幂比乘法和除法具有更高的优先级,进而也具有比加法和减法更高的优先级。例如,\verb|2 * 3 ^ 4|表示\verb|2 * (3 ^ 4)|,而\verb|2 + 3 * 4|表示\verb|2 + (3 * 4)|。此外,求幂运算是右结合的,而其他四种算术操作符则是左结合的。例如,\verb|2 ^ 3 ^ 4|意味着\verb|2 ^ (3 ^ 4)|,而\verb|2 - 3 + 4|则指的是\verb|(2 - 3) + 4|。但实际上,在算术表达式里显式使用括号通常比依靠上述惯例表达得更为清楚。 

除了操作整数的函数外,这个库文件还提供了一些有用的列表操作函数。在Haskell中,列表中的元素用方括号括上,并以逗号分隔。一些最常用的列表操作库函数说明如下:

\begin{itemize}
\item 从一个非空列表中选出第一个元素:

\noindent\hspace*{1cm} $> head~[1,~2,~3,~4,~5]$\\
\hspace*{1cm} $1$

\item 从一个非空列表中删除第一个元素:

\noindent\hspace*{1cm} $> tail~[1,~2,~3,~4,~5]$\\
\hspace*{1cm} $[2,~3,~4,~5]$

\item 选出列表中的第$n$个元素(从0开始计数):

\noindent\hspace*{1cm} $> [1,~2,~3,~4,~5]~!!~2$\\
\hspace*{1cm} $3$

\item 选出列表中的前$n$个元素:

\noindent\hspace*{1cm} $> take~3~[1,~2,~3,~4,~5]$\\
\hspace*{1cm} $[1,~2,~3]$

\item 从列表中删除前$n$个元素:

\noindent\hspace*{1cm} $> drop~3~[1,~2,~3,~4,~5]$\\
\hspace*{1cm} $[4,~5]$

\item 计算列表的长度:

\noindent\hspace*{1cm} $> length~[1,~2,~3,~4,~5]$\\
\hspace*{1cm} $5$

\item 计算数字列表中元素之和:

\noindent\hspace*{1cm} $> sum~[1,~2,~3,~4,~5]$\\
\hspace*{1cm} $15$

\item 计算数字列表中元素之积:

\noindent\hspace*{1cm} $> product~[1,~2,~3,~4,~5]$\\
\hspace*{1cm} $120$

\item 连接两个列表:

\noindent\hspace*{1cm} $> [1,~2,~3]~++~[4,~5]$\\
\hspace*{1cm} $[1,~2,~3,~4,~5]$

\item 反转列表:

\noindent\hspace*{1cm} $> reverse~[1,~2,~3,~4,~5]$\\
\hspace*{1cm} $[5,~4,~3,~2,~1]$

\end{itemize}

针对某些参数值,标准\textit{Prelude}中的一些函数可能产生错误。比如试图除零或从一个空列表中选择第一个元素都会产生一个错误:

\noindent\hspace*{1cm} $> 1~`div`~0$\\
\hspace*{1cm} $Error$\\
\hspace*{1cm} $> head [~]$\\
\hspace*{1cm} $Error$

实际上,当出现错误时,\textit{Hugs}系统也会产生一个消息,提供一些有关错误原因的信息。

作为参考,附录A介绍了标准Prelude最常用的一些定义,附录B显示了一些特殊的Haskell符号,如\verb|^|和$+$,以及如何使用键盘输入这些符号。

\section{函数应用}

在数学中,将函数应用于参数通常表示为用括号将参数括起来,且将两值相乘往往采用习惯性的表示,将两个值一个接着一个的写即可。例如,在数学中,表达式

\noindent\hspace*{1cm} $f~(a,~b)~+~c~d$

意为将函数$f$应用于两个参数$a$和$b$上,并将结果与$c$和$d$的乘积相加。为了反映函数在语言中的主流位置,在Haskell中函数程序习惯性地使用空格表示,而两值相乘则显式地使用$*$操作符表示。例如,上面的表达式使用Haskell编写如下:

\noindent\hspace*{1cm} $f~a~b + c~*~d$

此外,函数程序拥有比其它操作符更高的优先级。比如,$f~a + b$意为$(f~a) +
b$。下表给出了一些例子,来进一步说明函数记法在数学与Haskell之间的差异。

\begin{table}[htbp]
\label{tab:threesome}
\centering
\begin{tabular}{ll}
\hline
Mathematics & Haskell  \\
\hline
$f(x)$ & f~x \\
$f(x,y)$ & f~x~y \\
$f(g(x))$ & f~(g~x) \\
$f(x,g(y))$ & f~x~(g~y)\\
$f(x)g(y)$ & f~x~*~g~y\\
\hline
\end{tabular}
\end{table}

注意上面的表达式$f~(g~x)$在Haskell中依然需要括号,因为$f~g~
x$本身会被解释为将函数$f$应用于两个参数$g$和$x$。然而其本意却是将$f$应用于一个参数上,该参数即是将函数$g$应用于参数$x$上的结果。同样这个注意项也适用于表达式$f~x~(g~y)$.

\section{Haskell脚本}
除了标准Prelude所提供的函数外,你也可以定义新的函数。你无法在Hugs的>提示符下定义新函数,只能在\textit{脚本}中定义。脚本是一个由一系列定义组成的文本文件。按照惯例,Haskell脚本通常用\textit{.hs}作为文件后缀名以区别于其他种类的文件。

\subsection{我的第一个脚本}
当开发一个Haskell脚本时,保持两个窗口一直打开着是很有用的:一个窗口运行脚本的编辑器,另外一个运行Hugs。举个例子,假设我们启动文本编辑器输入两个函数的定义,并保存脚本到一个名为\textit{test.hs}的文件中:

\noindent\hspace*{1cm} $double~x~= x + x$\\
\hspace*{1cm} $quadruple~x~= double~(double~x)$

相应的,假设我们保持编辑器窗口处于打开状态,而在另外一个窗口中启动Hugs并输入指令使其加载这个新脚本:

\noindent\hspace*{1cm} $> :load test.hs$

现在\textit{Prelude.hs}和\textit{test.hs}都已被加载,两个脚本中的函数都可以自由使用了。比如:

\noindent\hspace*{1cm} $> quadruple~10$\\
\hspace*{1cm} $40$\\
\hspace*{1cm} $> take~(double~2)~[1,~2,~3,~4,~5,~6]$\\
\hspace*{1cm} $[1,~2,~3,~4]$

现在保持Hugs处于启动状态,我们返回到编辑器窗口。将下面两个函数的定义添加到脚本中,并重新保存文件。

\noindent\hspace*{1cm} $factorial~n = product~[~1..n~]$\\
\hspace*{1cm} $average~ns = sum~ns~`div`~length~ns$

我们同样可以这样定义:$average~ns = div~(sum~ns)~(length~ns)$,
但是将\textit{div}放在两个参数中间更加自然。一般情况下,任何接受两个参数的函数都可以写成将函数名用反单引号(`)括上后放在其参数之间的形式。

当脚本被修改后,Hugs不会自动加载它们,所以在使用新定义之前必须执行一个reload命令:

\noindent\hspace*{1cm} $> :reload$\\
\hspace*{1cm} $> factorial~10$\\
\hspace*{1cm} $3628800$\\
\hspace*{1cm} $> average~[1,~2,~3,~4,~5]$\\
\hspace*{1cm} $3$

作为参考,下表总结了一些Hugs中最常用命令的含义。请注意,每条命令都可以通过它的第一个字符进行缩写。例如,$:load$可以缩写为$:l$。命令$:type$将在后面的篇章中详细解释。

\begin{table}[htbp]
\label{tab:threesome}
\centering
\begin{tabular}{ll}
\hline
命令 & 含义\\
\hline
\textit{:load name} & 加载脚本\textit{name} \\
\textit{:reload} & 重新加载当前脚本 \\
\textit{:edit name} & 编辑脚本\textit{name} \\
\textit{:type expr} & 显示\textit{expr}的类型信息 \\
\textit{:?} & 显示所有命令 \\
\textit{:quit} & 退出Hugs \\
\hline
\end{tabular}
\end{table}

\subsection{命名需求}
定义一个新函数时,函数以及其参数的名字必须以小写字母开头,但后面可以跟随零个或多个字母(包括小写和大写),数字,下划线和正向单引号。例如,以下名字都是合法的:

\centerline{$myFun~~~fun1~~~arg~2~~~x’$}

下面列表中的\textit{关键字}在语言中都有着特殊的含义,并且不能作为函数或其参数的名字使用:

\begin{center}
\textbf{case~~class~~data~~default~~deriving~~do~~else}\\
\textbf{if~~import~~in~~infix~~infixl~~infixr~~instance}\\
\textbf{let~~module~~newtype~~of~~then~~type~~where}
\end{center}

按照惯例,在Haskell中列表参数的名字中通常有一个后缀\textit{s},表明它们可能含有多个值。例如,一个数字列表可能被命名为\textit{ns},一个包含任意值的列表可能会被命名为\textit{xs},一个字符列表可能被命名的\textit{css}。

\subsection{布局规则}

在一个脚本中,每个定义必须精确的从相同的列开始。这种\textit{布局规则}使我们能够根据代码缩进确定定义分组。例如脚本:

\noindent\hspace*{1cm} $a = b + c$\\
\hspace*{2cm} $        where $\\
\hspace*{3cm} $            b = 1$\\
\hspace*{3cm} $            c = 2$\\
\hspace*{1cm} $d = a * 2$

通过缩进可以很清楚的看出$b$和$c$是在$a$定义体中使用的局部定义。如果需要,这个分组可以显式的通过花括号将一系列定义括起来,并且定义之间可以用分号隔开。例如,上面的脚本也可以写成:

\noindent\hspace*{1cm} $a = b + c$\\
\hspace*{2cm} $        where $\\
\hspace*{3cm} $            \{b = 1;$\\
\hspace*{3cm} $            c = 2\}$\\
\hspace*{1cm} $d = a * 2$

但一般来说,依赖布局规则来确定定义分组比使用显式语法更加清晰。

\subsection{注释}

除了新的定义,脚本也会包含注释,但注释将被Hugs忽略。Haskell提供了两种类型的注释,分别称为\textit{普通注释}和\textit{嵌套注释}。普通注释以符号-开始,作用延伸到当前行的结尾,如下面的例子所示:

\noindent\hspace*{1cm} — Factorial of a positive integer:\\
\hspace*{1cm} $factorial~n~=~product~[1 . . n ]$\\
\hspace*{1cm} — Average of a list of integers:\\
\hspace*{1cm} $average~ns~= sum~ns~`div`~length~ns$

嵌套注释的开始和结束符号为\{-和-\},嵌套注释可以跨多行,还可能包含其他注释。嵌套注释在临时删除脚本中的某段定义时特别有用,如下面的例子:

\noindent\hspace*{1cm} \{-\\
\hspace*{1cm} $double~x$\\
\hspace*{1cm} $quadruple~x$\\
\hspace*{1cm} -\}

\section{本章备注}
Hugs系统可从Haskell主页www.haskell.org上自由下载,另外Haskell主页上还提供了其他有用的资源。

\section{习题}

\begin{enumerate}
\item 用括号显式标出下面算术表达式的结合情况:

\noindent\hspace*{1cm} $2$~\verb|^|~$3~*~4$\\
\hspace*{1cm} $2~*~3~+~4~*~5$\\
\hspace*{1cm} $2~+~3~*~4$~\verb|^|~$5$

\item 使用Hugs执行一遍本章所提供的例子

\item
下面的脚本中包含三处语法错误,纠正这些错误并使用Hugs确认你的脚本可以正常工作。

\noindent\hspace*{1cm} $N~ = a~`div`~length~xs$\\
\hspace*{2cm} $       where$\\
\hspace*{3cm} $          a~=~10$\\
\hspace*{2.5cm} $        xs = [1,~2,~3,~4,~5]$

\item Show how the library function last that selects the last element of a
non-empty list could be defined in terms of the library functions introduced
in this chapter. Can you think of another possible definition?

\item Show how the library function \textit{init} that removes the last element from a
non-empty list could similarly be defined in two different ways.
\end{enumerate}
                                       % 第二章 第一步
\chapter{类型和类}
在这一章中,我们将介绍Haskell中两个最基本的概念:类型和类。我们首先解释什么是类型以及在Haskell中如何使用它,然后介绍一些基本类型以及使用这些基本类型构造更大类型的方法,详细讨论函数类型,最后介绍一下多态类型和类型类的概念。

\section{基本概念}
\textit{类型}是一组相关值的集合。例如,类型$Bool$包含两个逻辑值$False$和$True$,而类型$Bool
\rightarrow
Bool$则包含了所有将$Bool$类型参数映射为$Bool$类型结果的函数,如逻辑非函数$not$。我们使用记法$v::T$表示\textit{v}是类型\textit{T}的一个值,并且可以说\textit{v}“具有类型”\textit{T}。例如:

\noindent\hspace*{1cm} $False :: Bool$\\
\hspace*{1cm} $True :: Bool$\\
\hspace*{1cm} $not :: Bool \rightarrow Bool$

一般地说,符号::也可以用于尚未被求值的表达式,这种情况下,$e::T$意思是对表达式$e$求值将产生一个类型$T$的值。例如:

\noindent\hspace*{1cm} $not~False :: Bool$\\
\hspace*{1cm} $not~True :: Bool$\\
\hspace*{1cm} $not~(not~False):: Bool$

在Haskell中,每个表达式必须有一个类型,该类型通过一个先于表达式求值的过程计算得到,这个过程被称为\textit{类型推断}。这个过程的关键在于一个函数应用的类型规则,其中规定如果$f$是一个将$A$类型参数映射为$B$类型结果的函数,且$e$是一个类型$A$的表达式,那么$f~~e$具有类型$B$:\\

\begin{center}
$\dfrac{f::A \rightarrow B~~e::A}{f~e::B}$
\end{center}

例如,$not~False::Bool$可通过这样的规则推断,该规则使用了这样的事实:$not::Bool
\rightarrow
Bool$和$False::Bool$。另一方面,表达式$not~3$通过上述规则无法推断类型,因为这需要$3::Bool$,但$3$不是一个逻辑值,这是无效的。像$not~3$这样的表达式无法确定类型,也可以说成是包含了一个类型错误,被视为无效表达式。

由于类型推断在求值过程之前,所以Haskell程序是\textit{类型安全}的,也就是说在求值过程中不会发生类型错误。实际上,类型推断能检查出程序中所占错误比例较高的一类错误,它也是Haskell最有用的特点之一。但是注意使用类型推断并不能消除发生在求值阶段的其他类错误的可能性,比如,表达式$1~'div'~0$可以通过类型推断检查,但在求值阶段报错,因为被$0$除的行为是未定义的。

类型安全的不足之处在于一些求值阶段成功的表达式却因类型原因而被拒绝。例如,条件表达式\textbf{if} 
$True$ \textbf{then} $1$ \textbf{else} $False$求值结果为$1$,
但是因包含一个类型错误而被视为无效表达式。
特别是,条件表达式的类型推断规则要求所有可能的结果都具有相同的类型,而在这里例子中,第一种结果为$1$,是一个数字类型,而第二个结果是$False$,是一个逻辑值类型。在实践中,程序员很快就学会了如何在类型系统的限制下工作以及如何避免这些问题。

在Hugs系统中,任意表达式的类型都可以通过$:type$显示出来,例如:

\noindent\hspace*{1cm} $> :type~not$\\
\hspace*{1cm} $not :: Bool \rightarrow Bool$\\
\hspace*{1cm} \\
\hspace*{1cm} $> :type~not~False$\\
\hspace*{1cm} $not~False :: Bool$\\
\hspace*{1cm} \\ 
\hspace*{1cm} $> :type~not~3$\\
\hspace*{1cm} $Error$

\section{基本类型}
\noindent Haskell提供了很多内置到语言中的基本类型,其中最常用的类型描述如下:
\\
\\
$Bool$~-~ \textbf{逻辑值}\\
这个类型包含了两个逻辑值$False$和$True$。
\\
\\
$Char$~-~\textbf{字符}\\
这个类型包含了普通键盘上提供的所有单个字符,如'$a$','$A$','$3$'和\verb|'_'|,以及拥有特殊效果的控制字符,如\verb|'\n'|(移动到新的一行)和\verb|'\t'|(移动到下一个制表位)。正如其他大多数编程语言的标准一样,单个字符必须用单引号'~'括起。
\\
\\
$String$~-~\textbf{字符串}\\
这个类型包含了所有字符序列,诸如"$abc$","$1~+~2~=~3$"以及空字符串""。同样正如其他大多数编程语言的标准,字符串必须用双引号" "括起。
\\
\\
$Int$~-~\textbf{固定精度整数}\\
这个类型包含诸如$-100$,$0$以及$999$这样的整数,计算机以固定大小的内存存储这些值。例如,Hugs系统中$Int$类型的取值范围在$-2^{31}$和$2^{31}-1$之间。超出这个范围将得到非预期的结果。例如,在Hugs系统中对\verb|2^31::Int|(使用::将强制结果为$Int$类型而不是其他的数值类型)进行求值将得到一个负值,这是不正确的。
\\
\\
$Integer$~-~\textbf{任意精度整数}\\
该类型包含所有的整数,我们使用足够多的内存储存这个类型的值,从而避免了对该类型值的范围强加上限和下限。例如,使用任何Haskell系统对\verb|2^31::Integer|求值都可以得到正确的结果。

除了对内存和精度的需求不同外,在$Int$和$Integer$之间数字类型的选择还是性能考量之一。特别是,大多数电脑都内置了用来处理固定精度整数的硬件,而任意精度的整数通常必须被看成数字序列,通过速度较慢的软件来处理。
\\
\\
$Float$~-~\textbf{单精度浮点数}\\
这个类型包含带小数点的数字,诸如$-12.34$,$1.0$以及$3.14159$,计算机以固定大小内存存储这些值。浮点一词源于这样一个事实:即小数点后允许的数字位数取决于数的大小。例如使用Hugs对$sqrt
~2 ::
Float$求值结果为$1.41421$(库函数$sqrt$用于计算一个数的平方根),其中小数点后有五位数字;而对$sqrt~99999::Float$求值结果为$316.226$,小数点后则只有三位数字。采用浮点数编程是一个专家话题,需要认真对待舍入误差。在本书入门性的文字中,我们将很少说到这种类型。

最后,我们注意到一个单个数字可能拥有不止一种数值类型。例如,$3 :: Int$,$3 ::
Integer$和$3 ::
Float$对于数字$3$来说都是有效的类型。这就提出了一个有趣的问题:这些数字在类型推断过程中究竟应该被分配什么类型?这个问题将在本章后面考量类型类时回答。

\section{列表类型}
\textit{列表}是一个由相同类型元素组成的序列,其元素放在方括号中,并使用逗号分隔。我们将元素类型为$T$的所有\textit{列表}类型记作[$T$]。比如:

\noindent\hspace*{1cm} $[False,~True,~False]~::~[Bool]$\\
\hspace*{1cm} $[’a’,~’b’,~’c’,~’d’]~::~[Char]$\\
\hspace*{1cm} $["One",~"Two",~"Three"]~::~[String]$

一个列表中元素的个数称为列表的\textit{长度}。长度为0的列表$[~~]$称为空列表,而长度为1的列表,如$[False]$和$['a']$,称为单件列表。注意$[~[~~]~]$和$[~~]$是两个不同的列表,前者是一个单件列表,组成该列表的唯一的元素是一个空列表,而后者则仅仅是一个空列表。

关于列表类型这里有三点需进一步注意。首先,一个列表的类型中没有传达其长度信息。例如,$[False,~
True]$和$[False,~True,~
False]$两者都是$[Bool]$类型,即使他们的长度不同。其次,列表的元素没有类型限制。目前我们局限在我们所能给出的例子范围内,因为到目前为止我们介绍的唯一的非基本类型就是列表类型,但是我们可以定义由列表类型元素组成的列表,例如:

\noindent\hspace*{1cm} $[[’a’,~’b’],~[’c’,~’d’,~’e’]]~::~[[Char]]$

最后,对一个列表的长度没有任何限制。特别是,正如我们将在第12章看到的那样,由于在Haskell中使用了惰性求值,具有无限长度的列表是自然且实用的。

\section{元组类型}
\textit{元组}是一个由类型可能不同的元素组成的有限长度序列,其元素放在圆括号中,并使用逗号分隔。我们用($T_1,~T_2,~...,~T_n$)表示所有元组的类型。对于从$i$到$n$范围内的任意值$i$,第$i$个元素具有类型$T_i$。例如:

\noindent\hspace*{1cm} $(False,~True)~::~(Bool,~Bool)$\\
\hspace*{1cm} $(False,~’a’,~True)~::~(Bool,~Char,~Bool)$\\
\hspace*{1cm} $("Yes",~True,~’a’)~::~(String,~Bool,~Char)$

一个元组中元素的个数称为\textit{元数}(arity)。元数为0的元组
$(~)$称为空元组,元数为2的元组称为二元组,元数为3的元组称为三元组,等等。元数为1的元组,例如($False$),是不允许使用的,因为它们将与显式设置求值顺序的括号的使用相冲突,如表达式$(1~+~2)~*~3$。

与列表类型一样,关于元组类型也有三点需进一步注意。首先,元组的类型传达了元数信息。例如,类型($Bool,
~Char$)包含了所有第一个元素为$Bool$类型且第二个元素为$Char$类型的二元组。其次,元组中的元素没有类型限制。例如,我们可以定义由元组类型元素组成的元组,由列表类型元素组成的元组以及由元组类型元素组成的列表:

\noindent\hspace*{1cm} $(’a’,~(False,~’b’))~::~(Char,~(Bool,~Char))$\\
\hspace*{1cm} $([’a’,~’b’],~[False,~True])~::~([Char],~[Bool])$\\
\hspace*{1cm} $[(’a’,~False),~(’b’,~True)]~::~[(Char,~Bool)]$

最后,注意元组必须具有有限元数,以保证元组类型总是能在求值过程之前被计算出来。

\section{函数类型}
\textit{函数}是一个从一种类型参数到另外一种类型结果的映射。我们用$T_1
\rightarrow T_2$表示所有将$T_1$类型参数映射为$T_2$类型结果的函数。例如:

\noindent\hspace*{1cm} $not~::~Bool \rightarrow Bool$\\
\hspace*{1cm} $isDigit~::~Char \rightarrow Bool$

(库函数$isDigit$用于判断一个字符是否是一个数字位)由于对函数的参数类型和结果类型没有任何限制,使用单一参数和结果的函数的简单概念就足以应付多个参数和结果的情况,只需将多个值使用列表或元组打包即可。例如,我们下面定义函数$add$,用于计算一个整数二元组的元素之和;定义函数$zeroto$,用于返回一个从$0$到给定上限值的整数列表:

\noindent\hspace*{1cm} $add~::~(Int,~Int) \rightarrow Int$\\
\hspace*{1cm} $add~(x~,~y)~=~x~+~y$\\
\hspace*{1cm} $zeroto~::~Int \rightarrow [Int]$\\
\hspace*{1cm} $zeroto~n~=~[0~. .~n]$

在这些例子中我们遵循了Haskell将函数类型放在函数定义之前作为参考文档的惯例。系统将检查由用户手工提供的类型与类型推断自动计算出的类型两者之间的一致性。

注意没有限制要求函数对它们的参数类型必须要有预期结果。这样一来,对于函数的某些参数来说,其结果可能是未定义的。比如当列表为空时,库函数$head$从列表中选出第一个元素的行为就是未定义的。

\section{Curried函数}
函数可以自由地将函数作为结果返回,利用这个事实,接受多个参数的函数也可以使用另外一种也许并不显而易见的方式处理。例如,考虑下面的定义:

\noindent\hspace*{1cm} $add'~::~Int \rightarrow (Int \rightarrow Int)$\\
\hspace*{1cm} $add'~x~y~=~x~+~y$

函数类型表明$add'$是一个函数,其接受的参数类型为$Int$,返回结果为一个类型为$Int \rightarrow Int$的函数。定义本身表明$add'$接受一个整数$x$为参数,后面跟着一个整数$y$,返回结果为$x~+ ~y$。更确切地说,$add'$接受整数$x$为参数,返回一个接受整数$y$为参数且返回$x~+~y$的函数。

注意函数$add'$产生的最终结果与上一节中的函数$add$相同。然而函数$add$将其两个参数打包为一个二元组后一同处理,而函数$add'$则一次仅接受处理一个参数,正如这两个函数的类型所反映的那样:

\noindent\hspace*{1cm} $add~::~(Int,~Int) \rightarrow Int$\\
\hspace*{1cm} $add'~::~Int \rightarrow (Int \rightarrow Int)$

对于有两个以上参数的函数,也可以使用同样的技术处理,通过返回以函数为返回值的函数,等等。例如,函数$mult$接受三个参数,每次接受一个,并返回它们的乘积,其定义如下:

\noindent\hspace*{1cm} $mult~::~Int \rightarrow (Int \rightarrow (Int \rightarrow Int))$\\
\hspace*{1cm} $mult~x~y~z~=~x~*~y~*~z$

这个定义表明$mult$接受一个整数$x$为参数并返回一个函数,后者依次接受一个整数$y$为参数并返回另外一个函数,最后这个函数接受整数$z$为参数,并最终返回结果$x~*~y~*~z$。

诸如$add'$和$mult$这样的每次接受一个参数的函数被称为$curried$。除了本身有吸引力之外,curried函数也比接受元组作为参数的函数更加灵活,因为一些有用的函数通常可以通过\textit{部分应用}参数不完整的curried函数来实现。例如,一个完成递增功能的函数可以通过curried函数$add'$的部分应用$add'~1 :: Int \rightarrow Int$实现,后者仅需要两个参数中的一个。

为避免在使用curried函数工作时过度使用括号,我们采纳了两个简单的惯例。首先,类型中使用的箭头$\rightarrow$是右结合的,例如:

\noindent\hspace*{1cm} $Int \rightarrow Int \rightarrow Int \rightarrow Int$

意为

\noindent\hspace*{1cm} $Int \rightarrow (Int \rightarrow (Int \rightarrow Int))$

然而使用空格表示的函数应用则是左结合的,例如:

\noindent\hspace*{1cm} $mult~x~y~z$

意为

\noindent\hspace*{1cm} $((mult~x)~y)~z$

除非显式需要元组,Haskell中的所有接受多个参数的函数一般都会被定义成curried函数,并且使用上面的两个惯例来减少需要使用的括号的数量。

\section{多态类型}

库函数\textit{length}用于计算任意列表的长度,无论列表中的元素是什么类型的。比如,它可以用于计算整型列表、字符串型列表甚至是函数类型列表的长度:\\
\hspace*{1cm} > $length~[1,~3,~5,~7]$\\
\hspace*{1cm} $4$\\
\hspace*{1cm} > $length~["Yes",~"No"]$\\
\hspace*{1cm} $2$\\
\hspace*{1cm} > $length~[isDigit~,~isLower~,~isUpper]$\\
\hspace*{1cm} $3$

通过在类型中包含\textit{类型变量},使得将$length$应用于由任意类型元素组成的列表的想法得以精确实现。类型变量必须以小写字母开头,通常命名为\textit{a,b,c}等。例如,\textit{length}的类型如下:\\
\hspace*{1cm} $length :: [a] \rightarrow Int$

即对任意类型\textit{a},函数\textit{length}具有类型$[a]
\rightarrow
Int$。包含一个或多个类型变量的类型被称作\textit{多态的}(“多种形式"),使用这种类型的表达式也是多态的。因此$[a] \rightarrow
Int$是一个多态类型,\textit{length}是一个多态函数。更普遍的是,标准Prelude中提供的很多函数都是多态的,例如:\\
\hspace*{1cm} $fst~::~(a,~b) \rightarrow a$\\
\hspace*{1cm} $head~::~[a] \rightarrow a$\\
\hspace*{1cm} $take~::~Int \rightarrow [a] \rightarrow [a]$\\
\hspace*{1cm} $zip~::~[a] \rightarrow [b] \rightarrow [(a,~b)]$\\
\hspace*{1cm} $id~::~a \rightarrow a$

\section{重载类型}

算术运算符$+$用于计算任意两个相同数值类型数的和。例如,它可以用来计算两个整数的和或两个浮点数的和:

\noindent\hspace*{1cm} $> 1~+~2$\\
\hspace*{1cm} $> 3$\\

\noindent\hspace*{1cm} $> 1.1~+~2.2$\\
\hspace*{1cm} $> 3.3$

通过在类型中包含\textit{类约束},使得将$+$应用于任意数值类型数字的想法得以精确实现。 类约束写成$C~a$,其中$C$是类的名字,$a$是一个类型变量。例如,$+$的类型如下: 

\noindent\hspace*{1cm} $(+)~::~Num~a \Rightarrow  a \rightarrow a \rightarrow a$

即对于任意一个数值类型类$Num$的实例类型$a$,函数$(+)$具有类型$a \rightarrow a
\rightarrow
a$。(将操作符括起来将之转换为一个curried函数,下一章中将详细解释其中缘由。)一个包含一个或多个类约束的类型称为\textit{重载的},使用这种类型的表达式也是重载的。因此,$Num~a \Rightarrow  a \rightarrow a \rightarrow a$是一个重载类型,$(+)$是一个重载函数。更普遍的是,标准Prelude库中提供的大多数数值函数都是重载的,例如:

\noindent\hspace*{1cm} $(-)~::~Num~a \Rightarrow  a \rightarrow a \rightarrow a$\\
\hspace*{1cm} $(*)~::~Num~a \Rightarrow  a \rightarrow a \rightarrow a$\\
\hspace*{1cm} $(negate)~::~Num~a \Rightarrow  a \rightarrow a$\\
\hspace*{1cm} $(abs)~::~Num~a \Rightarrow  a \rightarrow a$\\
\hspace*{1cm} $(signum)~::~Num~a \Rightarrow  a \rightarrow a$

此外,数值本身也是重载的。例如,$3::Num~a \Rightarrow a$意为对于任意数值类型$a$,数字3具有类型$a$。

\section{基本类}
回顾一下,一个类型是一组相关值的集合。基于这个概念,一个\textit{类}是一组支持重载操作的类型的集合,这些重载操作被称为\textit{方法}。Haskell提供一定数量的内置的基本类,下面描述了其中最常用的类:
\\
\\
$Eq$~-~ \textbf{相等类型}\\
包含在这个类中的类型的值可以使用下面两个方法进行相等和不等比较:

\noindent\hspace*{1cm} $(==) :: a \rightarrow a \rightarrow Bool$\\
\hspace*{1cm} $(\neq) :: a \rightarrow a \rightarrow Bool$

所有的基本类型$Bool, Char, String, Int,
Integer$和$Float$都是Eq类的实例。列表和元组类型也是一样,如果它们的元素类型是Eq类的实例,例如:

\noindent\hspace*{1cm} $> False == False$\\
\hspace*{1cm} $True$\\

\noindent\hspace*{1cm} $>'a'~==~'b'$\\
\hspace*{1cm} $False$\\

\noindent\hspace*{1cm} $> "abc"~==~"abc"$\\
\hspace*{1cm} $True$\\

\noindent\hspace*{1cm} $> [1,~2]~==~[1,~2,~3]$\\
\hspace*{1cm} $False$\\

\noindent\hspace*{1cm} $> ('a',~False)~==~('a',~False)$\\
\hspace*{1cm} $True$\\

\noindent$Ord$~-~ \textbf{有序类型}\\
包含在这个类中的类型都是Eq类的实例,除此之外这些类型的值都是(线性)有序的,可以通过以下六个方法进行比较和处理:

\noindent\hspace*{1cm} $(<) :: a \rightarrow a \rightarrow Bool$\\
\hspace*{1cm} $(\leq) :: a \rightarrow a \rightarrow Bool$\\
\hspace*{1cm} $(>) :: a \rightarrow a \rightarrow Bool$\\
\hspace*{1cm} $(\geq) :: a \rightarrow a \rightarrow Bool$\\
\hspace*{1cm} $(min) :: a \rightarrow a \rightarrow a$\\
\hspace*{1cm} $(max) :: a \rightarrow a \rightarrow a$\\

所有的基本类型$Bool, Char, String, Int,
Integer$和$Float$都是\textit{Ord}类的实例。列表和元组类型也是一样,如果它们的元素类型是Ord类的实例,例如:

\noindent\hspace*{1cm} $> False~<~True$\\
\hspace*{1cm} $True$\\

\noindent\hspace*{1cm} $> min~'a'~'b'$\\
\hspace*{1cm} $'a'$\\

\noindent\hspace*{1cm} $> "elegant"~<~"elephant"$\\
\hspace*{1cm} $True$\\

\noindent\hspace*{1cm} $> [1,~2,~3]~<~[1,~2]$\\
\hspace*{1cm} $False$\\

\noindent\hspace*{1cm} $> ('a',~2)~<~('b',~1)$\\
\hspace*{1cm} $True$\\

\noindent\hspace*{1cm} $> ('a',~2)~<~('a',~1)$\\
\hspace*{1cm} $False$\\

请注意,字符串,列表和元组都是按字典序排序的;
也就是说与单词在字典中的顺序一样。例如,两个相同类型的二元组是有序的。如果它们的第一个元素是有序的,则无须考虑它们的第二个元素。或者如果它们的第一个元素相等,那么它们的第二个元素则必须是有序的。
\\
\\
\noindent$Show$~-~ \textbf{可显示的类型}\\
包含在这个类中的类型的值都可以通过下面方法转换为字符串:

\noindent\hspace*{1cm} $show :: a \rightarrow String$

所有的基本类型$Bool, Char, String, Int,
Integer$和$Float$都是\textit{Show}类的实例。列表和元组类型也是一样,如果它们的元素是Show类的实例,例如:

\noindent\hspace*{1cm} $> show~False$\\
\hspace*{1cm} "False"\\

\noindent\hspace*{1cm} > $show~'a'$\\
\hspace*{1cm} "~'$a$'~"\\

\noindent\hspace*{1cm} > $show~123$\\
\hspace*{1cm} "$123$"\\

\noindent\hspace*{1cm} > $show$ [1,~2,~3]\\
\hspace*{1cm} "$[1,~2,~3]$"\\

\noindent\hspace*{1cm} > show ($'a',~False$)\\
\hspace*{1cm} "$('a',~False)$"\\

\noindent$Read$~-~ \textbf{可读的类型}\\
这个类与类Show是一对,包含在这个类中的类型的值可以通过下面方法从字符串转换得到:

\noindent\hspace*{1cm} $read:: String \rightarrow a$

所有的基本类型$Bool, Char, String, Int,
Integer$和$Float$都是\textit{Read}类的实例。列表和元组类型也是一样,如果它们的元素是Read类的实例,例如:

\noindent\hspace*{1cm} $> read~"False"~::~Bool$\\
\hspace*{1cm} $False$\\

\noindent\hspace*{1cm} > $read~'a'~::~Char$\\
\hspace*{1cm} '$a$'\\

\noindent\hspace*{1cm} > $read~123~::~Int$\\
\hspace*{1cm} $123$\\

\noindent\hspace*{1cm} > $read$ "[1,~2,~3]"~::~[Int]\\
\hspace*{1cm} $[1,~2,~3]$\\

\noindent\hspace*{1cm} > $read$ "($'a',~False$)"\\
\hspace*{1cm} $('a',~False)$\\

例子中使用$::$决定结果的类型。然而在实际中,通常可通过上下文自动推断出必要的类型信息。例如,表达式$not~(read~
"False")$不需要显式的类型信息,因为使用逻辑非的$not$的程序暗示了$read~
"False"$必须具有$Bool$类型。

注意,如果参数不符合语法要求,那么$read$的结果将是未定义的。例如,表达式$not~(read~
"hello")$在求值时会产生一个错误,因为$"hello"$被$read$的结果不是一个逻辑值。
\\
\\
\noindent$Num$~-~ \textbf{数字类型}\\
包含在这个类中的类型都是$Eq$类和$Show$类的实例,除此之外这些类型的值都是数字,可以通过以下六个方法进行处理:


\noindent\hspace*{1cm} $(+)~::~a \rightarrow a \rightarrow a$\\
\hspace*{1cm} $(-)~::~a \rightarrow a \rightarrow a$\\
\hspace*{1cm} $(*)~::~a \rightarrow a \rightarrow a$\\
\hspace*{1cm} $(negate)~::~a \rightarrow  a$\\
\hspace*{1cm} $(abs)~::~a \rightarrow  a$\\
\hspace*{1cm} $(signum)~::~a \rightarrow a$

($negate$方法返回一个数的负数,$abs$返回绝对值,而$signum$则返回数的符号性。)
基本类型$Int, Integer$和$Float$都是Num类的实例,例如:

\noindent\hspace*{1cm} $> 1~+~2$\\
\hspace*{1cm} $3$\\

\noindent\hspace*{1cm} > $1.1~+~2.2$\\
\hspace*{1cm} $3.3$\\

\noindent\hspace*{1cm} > $negate~3.3$\\
\hspace*{1cm} $-3.3$\\

\noindent\hspace*{1cm} > $abs~(-3)$\\
\hspace*{1cm} $3$\\

\noindent\hspace*{1cm} > $signum~(-3)$\\
\hspace*{1cm} $3$\\

注意Num类没有提供除法,但是正如我们即将要看到的,Haskell提供两个特殊的类单独处理除法,一个类用于整数数字,而另外一个用于分数。
\\
\\
\noindent$Integral$~-~ \textbf{整数类型}\\
包含在这个类中的类型都是$Num$类的实例,除此之外这些类型的值都是整数,支持整数除法和整数取余:

\noindent\hspace*{1cm} $div~::~a \rightarrow a \rightarrow a$\\
\hspace*{1cm} $mod~::~a \rightarrow a \rightarrow a$\\

实际中,这两个方法经常写在其两个参数之间,并用单引号括上。基本类型$Int$和$Integer$是Integral类的实例。例如:

\noindent\hspace*{1cm} $> 7~'div'~2$\\
\hspace*{1cm} $3$\\

\noindent\hspace*{1cm} > $7~'mod'~2$\\
\hspace*{1cm} $1$\\

考虑到效率,一些标准Prelude库中既涉及列表又涉及整型的函数(如$length$,$take$和$drop$)被严格应用在$Int$这样的有限精度整数上,而不是应用到任意$Integral$类的实例上。如果需要,这些函数的通用版本在另外一个叫作$List.hs$的库中提供。
\\
\\
\noindent$fractional$~-~ \textbf{分数类型}\\
包含在这个类中的类型都是$Num$类的实例,但除此之外这些类型的值都是非整数,支持小数除法和小数倒数:

\noindent\hspace*{1cm} $(/)~::~a \rightarrow a \rightarrow a$\\
\hspace*{1cm} $(recip)~::~a \rightarrow a$\\

基本类型$Float$是分数类的一个实例,例如:

\noindent\hspace*{1cm} $> 7.0 / 2.0$\\
\hspace*{1cm} $3.5$\\
\\
\noindent\hspace*{1cm} $> recip~2.0$\\
\hspace*{1cm} $0.5$\\

\section{本章备注}
用术语$Bool$表示逻辑值类型是为了纪念\textbf{George
Boole}在符号逻辑领域做出的开创性贡献,而用术语$curried$表示函数一次只接受一个参数是用于纪念\textbf{Haskell
Curry}(Haskell语言本身也是以他命名的)在这类函数领域所做的工作。Haskell报告(25)中给出了关于类型系统的更详尽的描述,提供给专家的正式说明,可以在(20; 6)中找到。

\section{习题}

\begin{enumerate}
\item 下面的值是什么类型?

\noindent\hspace*{1cm} $[’a’, ’b’, ’c’]$\\
\hspace*{1cm} $(’a’, ’b’, ’c’)$\\
\hspace*{1cm} $[(False, ’O’), (True, ’1’)]$\\
\hspace*{1cm} $([False, True ], [’0’, ’1’])$\\
\hspace*{1cm} $[tail , init, reverse ]$

\item 下面的函数是什么类型?

\noindent\hspace*{1cm} $second~xs~=~head~(tail~xs)$\\
\hspace*{1cm} $swap~(x , y) = (y, x )$\\
\hspace*{1cm} $pair~x~y = (x , y)$\\
\hspace*{1cm} $double~x = x * 2$\\
\hspace*{1cm} $palindrome~xs = reverse~xs~==~xs$\\
\hspace*{1cm} $twice~f~x = f~(f~x )$

提示:如果函数定义中使用了重载操作符,注意包含必要的类约束。

\item 使用Hugs检查一下你关于前两个问题的回答

\item 为什么在一般情况下将函数类型都作为Eq类的实例是不可行的?什么时候是可行的?提示:两个类型相同的函数是相等的,如果两个类型相同的函数对于相等的参数始终返回相同的结果,那么这两个函数是相等的。

\end{enumerate}
                                 % 第三章 类型与类
\chapter{定义函数}
在本章中我们将介绍一些在Haskell中定义函数的机制。我们首先介绍条件表达式和守卫等式,然后介绍一种简单却强大的模式匹配思想,最后
介绍lambda表达式和段的概念。

\section{以旧造新}
也许定义新函数最直接的方法就是简单地将已有的一个或多个函数结合起来。例如,下面展示的一些库函数就是用这种方法定义的:

\begin{itemize}
\item 判断一个字符是否是数字

\hspace*{1cm} $isDigit~::~Char \rightarrow Bool$\\
\hspace*{1cm} $isDigit~c~=~c \geq '0'~~\&\&~~c \leq '9'$

\item 判断一个整数是否是偶数

\hspace*{1cm} $even~::~Integral~a \Rightarrow a \rightarrow Bool$\\
\hspace*{1cm} $even~c~=~n~`mod`~2 == 0$

\item 将一个列表在第$n$th个元素处拆分

\hspace*{1cm} $splitAt~::~Int \rightarrow [a] \rightarrow ([a],~[a])$\\
\hspace*{1cm} $splitAt~n~xs~=~(take~n~xs,~drop~n~xs)$

\item 倒数

\hspace*{1cm} $recip~::~Fractional~a \Rightarrow a \rightarrow a$\\
\hspace*{1cm} $recip~n~=~1~/~n$

\end{itemize}

注意上面$even$和$recip$类型中类约束的使用,精确的指明了这两个函数可以分别应用于任何整数类型和分数类型。

\section{条件表达式}
有一类函数,它们从许多种可能的结果中选择出一个最终结果,Haskell提供了很多不同的方式来定义这类函数。
最简单的方式就是使用\textit{条件表达式},
条件表达式使用被称为\textit{条件}的逻辑表达式在两个相同类型的结果中选出一个。如果条件为\textit{真},就选中第一个结果,否则选中第二个。例如:库函数$abs$的定义如下,该函数返回一个整数的绝对值:

\noindent\hspace*{1cm}$abs~::~Int \rightarrow Int$\\
\hspace*{1cm}$abs~n~=~\textbf{if}~n~ \geq ~0~ \textbf{then}~n~ \textbf{else} -n $

条件表达式可以嵌套,它们可以包含其他条件表达式。例如,库函数$signum$定义如下,它用来返回一个整型数的符号:

\noindent\hspace*{1cm}$signum~::~Int \rightarrow Int$\\
\hspace*{1cm}$signum~n~=~\textbf{if}~n~<~0~\textbf{then}~-1~\textbf{else} $\\
\hspace*{4cm}$\textbf{if}~n~==~\textbf{then}~0~\textbf{else}~1$

注意,与某些编程语言中的条件表达式不同,Haskell中的条件表达式必须包含\textbf{else}分支,这样就避免了众所周知的“else悬挂”问题。例如,如果\textbf{else}分支是可选的,那么表达式$\textbf{if}~True~\textbf{then~if}~False~\textbf{then}~1~\textbf{else}~2$既可以返回结果2,也可能产生一个错误,这取决于表达式中的\textbf{else}分支是被看作是内部条件表达式的一部分还是被看作是外部条件表达式的一部分。

\section{守卫等式}
作为条件表达式的一种替代方案,函数还可以使用\textit{守卫等式}来定义。在这类函数中,我们使用一些被称为\textit{守卫}的逻辑表达式从一些类型相同的结果中选择函数的最终结果。如果第一个守卫等式为\textit{真},那么第一个结果被选中;否则如果第二个守卫等式为\textit{真},则第二个结果被选中,依此类推。例如,库函数$abs$也可以以下面的方式定义: 

\begin{tabular}[t]{lll}
$abs~n~$&$|~n\geq 0$&$= ~ n$\\
&$|~otherwise$&$=~-n$\\
\end{tabular}

符号|读作“满足于,使得”。守卫$otherwise$在标准Prelude库文件中简单地定义为$otherwise~=~True$。虽然以$otherwise$作为一系列守卫的结尾不是必要的,但这样做为处理“所有其他情况”提供了一种便利的方式,同时也清楚地避免了因所有守卫都不为真而出错的情况。

较之条件表达式,使用守卫等式定义函数可读性更好。例如:使用如下守卫等式定义的库函数$signum$更容易理解。

\begin{tabular}[t]{lll}
$signum~n$&$|~n~<~0$&$=~-1$\\
&$|~n~==~0$&$=~0$\\
&$|~otherwise$&$=~1$\\
\end{tabular}

\section{模式匹配}

使用\textit{模式匹配}可以使许多函数拥有一个极为简单且直观的定义。在这类函数定义中,我们使用一些被称为\textit{模式}的语法表达式从一些类型相同的结果中选择出函数的最终结果。如果第一个模式匹配成功,那么第一个结果被选中;否则,如果第二个模式匹配成功,那么第二个结果被选中,依此类推。例如,库函数$not$的定义如下,它用来返回一个对逻辑值取非的结果:

\begin{tabular}[t]{lll}
$not$&::&$Bool \rightarrow Bool$\\
$not~False$&=&$True$\\
$not~True$&=&$False$\\
\end{tabular}

接受多个参数的函数也可以使用模式匹配定义,这种情况下,每个等式中各个参数的模式按顺序进行匹配。例如,库操作符$\&\&$定义如下,该操作符返回两个逻辑值与后的结果:

\begin{tabular}[t]{lll}
$(\&\&)$&::&$Bool \rightarrow Bool \rightarrow Bool$\\
$True~\&\&~True$&=&$True$\\
$True~\&\&~False$&=&$False$\\
$False~\&\&~True$&=&$False$\\
$False~\&\&~False$&=&$False$\\
\end{tabular}

然而,我们可以通过将最后三个等式合并为一个等式来简化这个函数的定义,合并后的等式使用可匹配任何值的\textit{通配符}模式$\_$,并返回与两个参数值无关的结果$False$。

\begin{tabular}[t]{lll}
$True~\&\&~True$&=&$True$\\
$\_~\&\&~\_$&=&$False$\\
\end{tabular}

根据第12章讨论的惰性求值,这个版本的函数定义还有这样的好处:如果第一个参数为$False$,那么我们可直接返回结果$False$,而无须对第二个参数进行求值。在实际中,标准库prelude使用了同样具有这个属性的等式定义$\&\&$。但只使用了第一个参数的值来选择哪个等式作为最终结果:

\begin{tabular}[t]{lll}
$True~\&\&~b$&=&$b$\\
$False~\&\&~\_$&=&$False$\\
\end{tabular}

即如果第一个参数为$True$,那么结果为第二个参数的值;如果第一个参数为$False$,那么结果就是$False$。

注意,因技术原因在一个等式中同样的名字不能被用于多个参数。例如,下面的操作符$\&\&$的定义就是基于这个观察:如果两个参数相等,那么结果也是同样的值,否则结果为$False$。但是由于上面的命名要求,这个定义是无效的:

\begin{tabular}[t]{lll}
$b~\&\&~b$&=&$b$\\
$\_~\&\&~\_$&=&$False$\\
\end{tabular}

然而如果需要,我们可以使用守卫等式来定义一个有效的版本,守卫等式用来判断两个参数是否相等:

\begin{tabular}[t]{lll}
$b~\&\&~c$ & $|~b~==~c$ &= $b$\\
& $|otherwise$ &= $False$\\
\end{tabular}

到目前为止,我们只考虑了基本模式,要么是值、要么是变量或是通配符模式。在本节其余部分,我们将介绍三种有用的将较小模式结合成较大模式的方法。

\subsection*{元组模式}
一个模式元组的本身就是一个模式,它可以匹配任何元数相同且所有元素都可按顺序匹配对应模式的元组。例如,库函数$fst$和$snd$定义如下,它们分别返回二元组的第一个和第二个元素:

\begin{tabular}[t]{lll}
$fst$&::&$(a,~b) \rightarrow a$\\
$fst~(x,~\_)$&=&$x$\\
\end{tabular}

\begin{tabular}[t]{lll}
$snd$&::&$(a,~b) \rightarrow b$\\
$snd~(\_,~y)$&=&$y$\\
\end{tabular}

\subsection*{列表模式}
同样,一个模式列表本身是一个模式,它可以匹配任何长度相同且所有元素都可按顺序匹配对应模式的列表。例如,用来判断一个列表是否精确的包含三个元素且第一个元素为'$a$'的函数$test$定义如下:

\begin{tabular}[t]{lll}
$test$&::&$[Char] \rightarrow Bool$\\
$test~['a',~\_,~\_]$&=&$True$\\
$test~\_$&=&$False$\\
\end{tabular}

到目前为止,我们一直将列表视为Haskell内置的原子类型。但事实上并非如此,它们实际上是使用操作符$:$从空列表[~]开始一次一个元素的构造起来的。操作符$:$被称为$cons$,它通过将一个新元素加到一个已存在列表的开始处来构造一个新的列表。例如,列表$[1,~2,~3]$可以按如下分解:

\noindent\hspace*{1cm} $[1,~2,~3]$\\
\hspace*{1cm} = \{列表记法\}\\
\hspace*{1cm} $1~:~[2,~3]$\\
\hspace*{1cm} = \{列表记法\}\\
\hspace*{1cm} $1~:~(2:~[3])$\\
\hspace*{1cm} = \{列表记法\}\\
\hspace*{1cm} $1~:~(2:~(3:~[]))$

即$[1,~2,~3]$只是$1~:~(2~:~(3~:~[~]))$的一个缩写。为了避免在使用这样的列表时过度使用括号,cons操作符被假定为是右结合的。例如,$1:~2:~3:~[~]$意为$1:~(2:~
(3:~ [~]))$。

cons操作符不仅可以用来构造列表,也可以用来构造模式,用于匹配任何非空且第一个元素及其余元素都可按顺序匹配对应模式的列表。例如,我们现在可以定义一个更通用的函数$test$的版本,用于判断包含任意数量字符的列表是否以'$a$'开头: 

\begin{tabular}[t]{lll}
$test$&::&$[Char] \rightarrow Bool$\\
$test~('a'~:~\_)$&=&$True$\\
$test~\_$&=&$False$\\
\end{tabular}

同样,库函数$null$,$head$和$tail$的定义如下,它们分别用于判断一个列表是否为空,从一个非空列表中选出第一个元素以及从一个非空列表中删除第一个元素:

\begin{tabular}[t]{lll}
$null$&::&$[a] \rightarrow Bool$\\
$null~[~]$&=&$True$\\
$null~(\_:\_)$&=&$False$\\
\end{tabular}

\begin{tabular}[t]{lll}
$head$&::&$[a] \rightarrow a$\\
$head~(x:\_)$&=&$x$\\
\end{tabular}

\begin{tabular}[t]{lll}
$tail$&::&$[a] \rightarrow a$\\
$tail~(\_:~xs)$&=&$xs$\\
\end{tabular}

注意cons操作符必须用括号括上,因为函数优先级高于所有其他操作符。例如,不带括号的定义$tail~\_:~xs~=~xs$意为$(tail~\_)~:~xs~=~xs$,它不仅含义不正确,而且还是一个无效的定义。

\subsection*{整数模式}
作为一种有时很有用的特殊情况,Haskell也允许$n + k$形式的整数模式\footnote{译注:$n+k$模式在Haskell 2010标准中默认被禁用。若使用$n+k$模式,在GHC 7及后续版本中必须开启NPlusKPatterns扩展。},其中$n$是一个整数变量,$k > 0$是一个整数常量。例如,函数$pred$定义如下,它将0映射到本身且将任何严格的整数映射为该整数的前驱:

\begin{tabular}[t]{lll}
$pred$&::&$Int \rightarrow Int$\\
$pred~0$&=&$0$\\
$pred~(n~+~1)$&=&$n$\\
\end{tabular}

关于$n~+~k$模式有两点需要注意。首先,它们只是匹配$\geq ~k$的整数。例如,对$pred~(-1)$求值将产生一个错误,因为$pred$定义中的两个模式都不匹配负数。其次,和cons模式一样,整数模式必须用括号括起来。例如,不带括号的定义$pred~n~+~1~=~n$意为$(pred~n)~+~1~=~n$,这是一个无效的定义。

\section{Lambda表达式}
作为使用等式定义函数的一种替代方案,函数还可以使用\textit{lambda表达式}来构造。lambda表达式包含一个用于参数匹配的模式以及一个详细说明了如何根据参数计算出结果的主体。lambda表达式定义的函数没有函数名,换句话说,lambda表达式是匿名函数。

例如,一个只接受一个数字为参数且返回结果$x~+~x$的匿名函数可以按下面方式构造:

\noindent\hspace*{1cm} $\lambda x \rightarrow x~+~x$

符号$\lambda$是小写希腊字母“lambda”。尽管lambda表达式构造的函数没有名字,但它们仍然可以像其他函数那样使用。例如:

\noindent\hspace*{1cm} $>~(\lambda x \rightarrow x~+~x)~~2$\\
\hspace*{1cm} $4$

除了自身有趣之外,lambda表达式还有很多实际应用。首先,它们可以用来形式化curried函数定义的含义。例如,定义

\noindent\hspace*{1cm} $add~x~y~=~x~+~y$

可以被理解为

\noindent\hspace*{1cm} $add~=~\lambda x \rightarrow (\lambda y \rightarrow x~+~y)$

这里精确地表明$add$是一个函数,它接受一个数字$x$作为参数且返回一个函数,而这个返回的函数依次接受数字$y$为参数且返回结果$x~+~y$。

其次,当定义返回值的本质为函数而不是curring结果的函数时,lambda表达式也非常有用。例如,库函数$const$定义如下,它返回一个总是产生给定值的常量函数:

\begin{tabular}[t]{lll}
$const$&::&$a \rightarrow b \rightarrow a$\\
$const~x~\_$&=&$x$\\
\end{tabular}

但是,更吸引人地定义$const$的方式是通过在类型中使用括号,在定义中使用lambda表达式以显式地指出用函数作为返回结果:

\begin{tabular}[t]{lll}
$const$&::&$a \rightarrow (b \rightarrow a)$\\
$const~x~$&=&$\lambda \_ \rightarrow x$\\
\end{tabular}

最后,lambda表达式可用于避免给仅被引用一次的函数命名。例如,函数$odds$定义如下,它用于返回前$n$个奇数:

\begin{tabular}[t]{lll}
$odds$&::&$Int \rightarrow [Int]$\\
$odds~n$&=&$map~f~[0..n-1]$\\
& &\textbf{where}~$~f~x~=~x~*~2~+~1$\\
\end{tabular}

(库函数$map$用于将一个函数应用于一个列表中的所用元素。)但是,由于本地定义的函数$f$只被引用一次,$odds$的定义可以使用lambda表达式简化为:

\noindent\hspace*{1cm} $odds~n~=~map~(\lambda x \rightarrow x~*~2~+~1)~[0..n-1]$

\section{段}

(⊕)

                                % 第四章 定义函数
\chapter{List comprehensions}

在本章节中,我们将...
                               % 第五章 List comprehensions

\end{document}

%
% } body end
%
