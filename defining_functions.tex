\chapter{定义函数}
在本章中我们将介绍一些在Haskell中定义函数的机制。我们首先介绍条件表达式和守卫等式,然后介绍一种简单却强大的模式匹配思想,最后
介绍lambda表达式和段的概念。

\section{以旧造新}
也许定义新函数最直接的方法就是简单地将已有的一个或多个函数结合起来。例如,下面展示的一些库函数就是用这种方法定义的:

\begin{itemize}
\item 判断一个字符是否是数字

\hspace*{1cm} $isDigit~::~Char \rightarrow Bool$\\
\hspace*{1cm} $isDigit~c~=~c \geq '0'~~\&\&~~c \leq '9'$

\item 判断一个整数是否是偶数

\hspace*{1cm} $even~::~Integral~a \Rightarrow a \rightarrow Bool$\\
\hspace*{1cm} $even~c~=~n~`mod`~2 == 0$

\item 将一个列表在第$n$th个元素处拆分

\hspace*{1cm} $splitAt~::~Int \rightarrow [a] \rightarrow ([a],~[a])$\\
\hspace*{1cm} $splitAt~n~xs~=~(take~n~xs,~drop~n~xs)$

\item 倒数

\hspace*{1cm} $recip~::~Fractional~a \Rightarrow a \rightarrow a$\\
\hspace*{1cm} $recip~n~=~1~/~n$

\end{itemize}

注意上面$even$和$recip$类型中类约束的使用,精确的指明了这两个函数可以分别应用于任何整数类型和分数类型。

\section{条件表达式}
假设某个函数,该函数从许多可能的结果中做出选择。Haskell提供了很多不同的方式来定义这类函数。 最简单的方式就是使用\textit{条件表达式},顾名思义, 它使用我们称为\textit{条件}的逻辑表达式,在两个相同类型的结果中选出一个。如果条件为{\it{真}},选中第一个结果,否则选中第二个。例如:库函数$abs$的定义如下,该函数返回一个整数的绝对值

\noindent\hspace*{1cm}$abs~::~Int \rightarrow Int$\\
\hspace*{1cm}$abs~n~=~\textbf{if}~n~ \geq ~0~ \textbf{then}~n~ \textbf{else} -n $

条件表达式可以嵌套,它的其中一个结果可以是另外一个条件表达式,举例来说,符号函数$signum$定义如下:

\noindent\hspace*{1cm}$signum~::~Int \rightarrow Int$\\
\hspace*{1cm}$signum~n~=~\textbf{if}~n~<~0~\textbf{then}~-1~\textbf{else} $\\
\hspace*{4cm}$\textbf{if}~n~==~\textbf{then}~0~\textbf{else}~1$

注意,不同于某些编程语言,Haskell中的条件表达式必须总是存在{\bf{else}}分支,这样就避免了众所周知的“else悬挂”问题。如果\textbf{else}分支是可选的,下面的表达式:

$\textbf{if}~True~\textbf{then~if}~False~\textbf{then}~1~\textbf{else}~2$

\textbf{else}分支既可以看做内部条件表达式的一部分,这样会返回结果2,也可以看做外部条件表达式的一部分,这样会产生一个错误。
\section{守卫等式}
作为条件表达式式的备选方案,函数还可以通过\textit{守卫等式}的方式来定义。在这种方式中,我们定义一系列相同类型的、有序的逻辑表达式,并称之为\textit{守卫},然后在这些守卫中选择我们的结果。我们首先计算第一个守卫,如果为\textit{真},第一个结果被选中,否则计算第二个守卫,如果为真,第二个结果被选中,依次类推。比如:库函数$abs$还可以以下面的方式定义。

\begin{tabular}[t]{lll}
$abs~n~$&$|~n\geq 0$&$= ~ n$\\
&$|~otherwise$&$=~-n$\\
\end{tabular}

符号|可读作“满足于,使得”。最后一个守卫$otherwise$仅仅是在库文件定义为$otherwise~=~True$,虽然不必使用$otherwise$最为一系列守卫的结尾,但这样做为处理“所有其他条件”提供了便利,同时,前面的守卫可能都不为真,这种情况下如果没有$otherwise$,将会产生一个错误。

较之条件表达式,守卫等式更具有易读性,例如:使用守卫等式定义的库函数$signum$更容易理解。


\begin{tabular}[t]{lll}
$signum~n$&$|~n~<~0$&$=~-1$\\
&$|~n~==~0$&$=~0$\\
&$|~otherwise$&$=~1$\\
\end{tabular}

