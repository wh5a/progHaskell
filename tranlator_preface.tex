%\XeTeXinputencoding "GBK"                                   % 本文件采用GBK编码

\chapter*{译者序}

\begin{quotation}
"A language that doesn't affect the way you think about programming, is not worth knowing".
\begin{flushright}
\textit{-- Alan Perlis(ACM第一任主席,图灵奖得主,1922-1990)}
\end{flushright}
\end{quotation}

《程序员修炼之道》一书作者建议程序员每年应至少学习一门新的语言。今年我选择了函数式编程语言Haskell。选择Haskell的理由正如Alan
Perlis所说的那样,Haskell是一门可以影响我的编程思维的语言。

开始接触Haskell后,我才发现它在国内是如此的小众(其实在国际上也很小众),国内居然没有正式出版过Haskell相关的中文书籍\footnote{据说Real world Haskell正在被翻译中},唯一可参考的就是网上流传的一本免费的由乔海燕翻译的《Yet Another Haskell Tutorial》,国内出版的原版书籍似乎也只有《Real World Haskell》这一本。

我开始学习Haskell时用的是《Real World Haskell》原版,书很厚,后来发现似乎不适合初学者。随后又在网上搜索资料,找到了Graham
Hutton编写的《Programming in Haskell》\footnote{\url{http://www.cs.nott.ac.uk/~gmh/book.html}}这本教程。《Programming in
Haskell》这本书很薄,加起来不到200页,而且在这本书的官方主页上可下载到与书配套的幻灯片资料和习题答案,非常适合Haskell初学者。

同样是在这本书的主页上,我发现这本书在2009年出版了日文版和韩文版,这个让我很是受触动,中文版怎么可以落后日韩呢?要不我就尝试翻译翻译吧。于是在Google
code上建立了《Programming in
Haskell》中文版的项目\footnote{\url{http://code.google.com/p/programming-in-haskell-cn}}。

自己边学Haskell,边尝试翻译了《Programming in
haskell》的前三章(这期间为了排版还学习了\TeX)。由于自己之前没有函数式编程语言的知识和经验,翻译起来很是吃力。另外Graham
Hutton的偏学术派的写作风格也让翻译的难度陡增不少。

经过这段时间的翻译,对Haskell以及函数式编程的理解也加深了不少,后续计划对已翻译的三个章节进行回顾,形成中英文对照表,纠正翻译错误以及行文不通顺的地方。
