%\XeTeXinputencoding "GBK"                                   % 本文件采用GBK编码

\chapter*{译者序}

\begin{quotation}
"A language that doesn't affect the way you think about programming, is not worth knowing".
\begin{flushright}
\textit{-- Alan Perlis(ACM第一任主席,图灵奖得主,1922-1990)}
\end{flushright}
\end{quotation}

《程序员修炼之道》一书作者建议程序员每年应至少学习一门新的语言,以拓宽思维,避免墨守成规。今年我选择了函数式编程语言
Haskell。选择Haskell的理由正如Alan
Perlis所说的那样,Haskell是一门可以影响程序员编程思维的语言,我也期望通过学习Haskell来拓宽我的思维。

开始接触Haskell后,我才发现它在国内是如此的小众(其实在国际上也很小众),国内居然没有正式出版过Haskell相关的中文书籍\footnote{据说Real
world Haskell正在被翻译中},唯一可参考的像样的中文资料就是网上流传的一本免费的由乔海燕翻译的《Yet
Another Haskell Tutorial》
,国内出版的影印版书籍似乎也只有《真实世界的Haskell》(英文名:Real World
Haskell)这一本。

我开始学习Haskell时用的就是那本曾获得过Jolt Award大奖的《Real World
Haskell》影印版,书很厚,是本Haskell大全。但后来发现似乎不太适合初学者。随后又在网上搜索资料,找到了Graham
Hutton 编写的《Programming in
Haskell》\footnote{\url{http://www.cs.nott.ac.uk/~gmh/book.html}}这本教程。与《Real World
Haskell》比起来,《Programming in
Haskell》这本书就显得“单薄”了许多,加起来总共不到200
页。不过这本书却非常适合函数式编程和Haskell的初学者,因为这本书是基于英国诺丁汉大学课程讲义编制而成,经过了多年实际教学检验,并且在这本书的官方主页上还可以下载到与书配套的讲义幻灯片和习题答案。

同样是也是在这本书的主页上,我发现了这本书在2009年就已经出版了日文版和韩文版,这个让我很是受触动,为什么在好书引进方面我们也落后于日韩呢!突然脑中迸发出一个念头:要不我来试试翻译一下这本书,也算是为Haskell在中国的发展做出一些自己的贡献。

于是在Google Code上建立了这个《Programming in Haskell》中文版翻译项目\footnote{\url{http://code.google.com/p/programming-in-haskell-cn}}。

真诚的欢迎大家提出建议和意见,帮助我来审校翻译中存在的问题,共同完成这个项目。

另外这里需声明一点:自己仅是一个Haskell爱好者和初学者,非Haskell牛人。请大家读译稿后谨慎拍砖!

\hspace*{12cm} Tony Bai\\
\hspace*{12cm} October, 2010
